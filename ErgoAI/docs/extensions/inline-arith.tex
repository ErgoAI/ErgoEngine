For instance,
%% 
\begin{verbatim}
   ?-  ?X \is count([a,b,c]).  // ?X = 3
   ?-  ?X \is avg([1,2,3]).    // ?X = 2
   ?-  ?X \is sum([1,2,3]).    // ?X = 6
   ?-  ?X \is nth(2,[a,b,c]).  // ?X = b
\end{verbatim}
%% 

  \index{inline arithmetic expression}
  \index{arithmetic expression!inline}
\paragraph{Inline evaluation of arithmetic expressions.}
In addition to the various extensions of
\texttt{\bs{}is/2}, \ERGO provides a more natural and powerful way to
evaluate expressions by placing them directly as arguments of the
predicates and frames, and prefixing them with a ``\texttt{=}'' or with
``\texttt{\bs{}is}''.
This can be done both in rule heads and rule bodies. For example,
%% 
\begin{alltt}
    q(2).\\
    p(=?X+3) :- q(?X).  // alternatively: p(\bs{}is ?X+3) :- q(?X). \\
    ?- p(?X).\\

    ?X = 5
\end{alltt}
%% 
Note that if we used the rule \texttt{p(?X+3) :- q(?X)} instead (i.e.,
without the prefix ``\texttt{=}'' or ``\texttt{\bs{}is}'') then the answer
to the above query would have been
\texttt{?X = 2+3} because arithmetic expressions are not evaluated by
default.

The following example illustrates the inline evaluation feature in a query
(it works identically in rule bodies):
%% 
\begin{alltt}
   q(2), p(5).\\
   ?- q(?X), p(=?X+3).  //  alternatively: ?- q(?X), p(\bs{}is ?X+3).  \\

   Yes
\end{alltt}
%% 
Perhaps the most interesting way to use this feature is in combination with
the string concatenation operator introduced earlier. This combination
provides Java-like printing facility as follows:
%% 
\begin{alltt}
   p(a,b).\\
   ?- p(?X,?Y), writeln(= result ||': '|| ?X|| + || ?Y)@\bs{}io.\\
   \textit{result: a+b}   
\end{alltt}
%% 
Observe that \emph{all}  
arithmetic functions, as well as list/string
concatenation, list operations, etc., are allowed in inline
expressions. For instance,
%% 
\begin{alltt}
   p(=sin(5)+atan(3)).\\
   ?- p(?X).\\
   {\it Resut\/}: ?X = 0.2901\\

   q(a,b), r([1,2,3],[1,3],[3,8]).\\
   p(=?X||?Y, \bs{}is ?Z++?V-\,-?W) :- q(?X,?Y), r(?Z,?V,?W).\\
   ?- p(ab,[1,2,3,1]).\\
   Yes
\end{alltt}
%% 

\noindent
\textbf{Note}:  Aggregate operators described in
Section~\ref{sec-aggregates} and user-defined functions (UDFs) described in
Section~\ref{sec:udf} are \emph{always} evaluated when encountered in an
argument position, so they never need to be prefixed with an ``='' or an
\texttt{\bs{}is}. For instance,
%% 
\begin{verbatim}
   \udf f(?X) := =?X+5 \if p(?X).
   p({1,2,3}).
   ?- insert{q(6)}.
   ?- insert{r(=f(?X))}.   // UDF f/1 used inline without a =
   ?- q(sum{?V|p(?V)}).   // Yes. Aggr sum{?V|p(?V)} used inline without a =
   ?- r(?Z).              // ?X = 6,7,8
\end{verbatim}
%% 

%%% Local Variables: 
%%% mode: latex
%%% TeX-master: "../ergo-manual"
%%% End: 
