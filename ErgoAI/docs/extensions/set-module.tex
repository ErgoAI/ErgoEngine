
\subsection{Sets, Maps, and Dictionaries}\label{sec-data-types-set}

\index{set}
\index{map}
\index{associative array}
\index{dictionary}
%%
This section introduces the built-in \FLSYSTEM module, called
\texttt{\bs{}set}, which provides two data structures:
%% 
\begin{itemize}
\item  Sets: a data structure that provides containers for terms and
  supports 
  %% 
  \begin{itemize}
  \item   efficient  membership search, whose complexity is independent on
    the size of the set (this is a big difference compared to lits)
  \item   duplicate elimination
  \end{itemize}
  %% 
\item  Single-valued key-value pairs: a data structure that efficiently associates search
  keys with values, where each key is associated with at most one
  value. This data structure is known as a \emph{map} in Java,
  \emph{dictionary} in Python, and \emph{associative array} in Perl.   
  For brevity, we shall use the term \emph{map} for this structure.
\index{mv-map}
\index{multi-valued map}
\item  Multi-valued key-value pairs: a map allows one to associate at most one
  value with each key. \FLSYSTEM makes it possible to have map-like
  structures that permits association of multiple values with any key.
  We will call this data structure a \emph{multi-valued map},
  of \emph{mv-map}. 
\end{itemize}
%% 

A set or a map is identified via user-selected abstract symbol (a Prolog atom)
and is created the first time something is inserted into that data structure.
The set/map identifier is \emph{global}, i.e., it does not need to be saved
somewhere between uses in different rules. 
For example,
%% 
\begin{verbatim}
    ?- myset123[insert([foo,bar,p(234)])]@\set.
\end{verbatim}
%% 
creates a set identified by the symbol \texttt{myset123} and inserts 
three elements into it: \texttt{foo}, \texttt{bar}, and \texttt{p(234)}.   
A set or a map exists until it is destroyed via
%% 
\begin{verbatim}
    ?- myset123[destroy]@\set.
\end{verbatim}
%% 
It is a good idea to destroy sets and maps that are no longer needed, to
free up space (these structures are not garbage-collected by the system).

There is no significant difference between the data structures used for
sets, maps,
and mv-maps. The only real difference lies in the APIs, i.e., the
operations that are used to work with sets and maps. Because of this affinity,
many API operations are common to
sets, maps, and mv-maps.

\subsubsection{API for Working with Sets, Maps, and MV-Maps}
\label{sec-api-set-map}

%% 
\begin{itemize}
\item  \texttt{?SetMap[exists]@\bs{}set} --- true if \texttt{?SetMap} is bound to
  a set/map/mv-map that was created and not destroyed.
  For example, \texttt{set1[exists]@\bs{}set}. 
\item \texttt{?SetMap[empty]@\bs{}set} --- true if \texttt{?SetMap} is bound to
  a set/map/mv-map that is empty or non-existent.
\item \texttt{?SetMap[destroy]@\bs{}set} --- destroys the set/map/mv-map that is
  identified via the atom bound to \texttt{?SetMap}.
\item \texttt{?SetMap[union(?SetMap2)->?SetMap3]@\bs{}set} --- \texttt{?SetMap3}
  becomes the union of \texttt{?SetMap} and \texttt{?SetMap2}. The variables
  \texttt{?SetMap}, \texttt{?SetMap2}, \texttt{?SetMap3} must be bound to symbols,
  which do not need to be distinct.
  If, for example, \texttt{?SetMap2} and \texttt{?SetMap3} are bound to the same
  symbol, \texttt{abc}, and \texttt{?SetMap} is bound to \texttt{foo},
  i.e., \texttt{foo[union(abc)->abc]@\bs{}set}, then the
  union of the sets/maps identified by \texttt{abc} and \texttt{foo} is
  computed and gets associated (destructively) with the symbol
  \texttt{abc}.

  For example, \texttt{set1[union(set2)->set3]@\bs{}set}. 

  \index{++ operator}
  This expression has a shortcut:  \texttt{set3 \bs{}is set1++set2}.   

\item \texttt{?SetMap[minus(?SetMap2)->?SetMap3]@\bs{}set} --- \texttt{?SetMap3}
  becomes associated with the set /map/mv-map that is a set-difference of \texttt{?SetMap}
  and \texttt{?SetMap2}. All three arguments must be bound to symbols, and
  \texttt{?SetMap2} may be bound to the same symbol as \texttt{?SetMap3} or
  \texttt{?SetMap}. For instance, \texttt{abc[minus(foo)->foo]@\bs{}set}      
  computes the difference of the set/map/mv-map identified by \texttt{abc} and the
  set/map/mv-map identified by \texttt{foo} and then \texttt{foo} becomes associated
  with that computed set/map/mv-map.

  Example: \texttt{set1[minus(set2)->set3]@\bs{}set}. 

  \index{-- operator}
  The expression \texttt{set1[minus(set2)->set3]@\bs{}set} has a shortcut:  \texttt{set3 \bs{}is set1--set2}.   

  \item \texttt{?SetMap[intersect(?SetMap2)->?SetMap3]@\bs{}set} --- \texttt{?SetMap3}
  becomes associated with the set/map/mv-map that is a set-intersection of \texttt{?SetMap}
  and \texttt{?SetMap2}. All three arguments must be bound to symbols, and
  neither \texttt{?SetMap} nor
  \texttt{?SetMap2} can be bound to the same symbol as
  \texttt{?SetMap3}.

  Example: \texttt{set1[intersect(set2)->set3]@\bs{}set}. 

  \index{\&\& operator}
  The expression \texttt{set1[intersect(set2)->set3]@\bs{}set} has a shortcut:  \texttt{set3 \bs{}is set1\&\&set2}. 

  \index{\bs{}subset operator}
  \index{operator!\bs{}subset}
\item \texttt{?SetMap[subset(?SetMap2)]@\bs{}set} --- true if \texttt{?SetMap}
  identifies a set that is a subset/submap of the set/map/mv-map identified by
  \texttt{?SetMap2}.
  \\
  \ERGO provides a convenient shortcut for this method: \texttt{?SetMap
  \bs{}subset ?SetMap2}.  
\item \texttt{?SetMap[equal(?SetMap2)]@\bs{}set} --- true if \texttt{?SetMap}
  identifies a set that is equal to the set/map/mv-map identified by \texttt{?SetMap2}.
\item \texttt{?SetMap[copy->?SetMap2]@\bs{}set} --- \texttt{?SetMap2} becomes
  associated with a copy of the set/map/mv-map identified by \texttt{?SetMap}.  
\item \texttt{?SetMap[tolist->?List]@\bs{}set} --- binds \texttt{?List} to
  the sorted list of elements in the set/map/mv-map identified by
  \texttt{?SetMap}. In case of maps and mv-maps, the members of the
  list have the form \texttt{key=value}.  For a set, the members in
  the list are simply the elements of the set.
\item \texttt{?SetMap[type->?Type]@\bs{}set} --- binds \texttt{?Type} to
  the type of the structure indetified by \texttt{?SetMap}. The possible
  values are \texttt{set}, \texttt{map},    and \texttt{mvm} (for mv-maps). 
\end{itemize}
%% 

\subsubsection{API for Working with Sets Only} \label{sec-api-set}

%% 
\begin{itemize}
\item  \texttt{?Set[insert(?Element)]@\bs{}set} --- \texttt{?Element} gets
  inserted in \texttt{?Set}. If \texttt{?Element} exists in the set already,
  the insertion simply succeeds without changing the set.
  If \texttt{?Element} is a list, all the elements in the list are inserted
  (except those that are already in \texttt{?Set}).  

  Note that \texttt{?Element} can have variables. 
  For example, \texttt{set123[insert([p(a),q(1,?X)])]@\bs{}set}. 
\item \texttt{?Set[delete(?Element)]@\bs{}set} --- \texttt{?Element} gets
  deleted from \texttt{?Set}. If \texttt{?Element} is not in the set,
  the deletion simply succeeds without changing the set.
  If \texttt{?Element} is a list, all the elements in the list are deleted
  (except those that were not in \texttt{?Set} to begin with). 
  
  Note that \texttt{?Element} can have variables. In that case,
  \texttt{?Element} will be non-deterministically
  unified with \emph{some} member of the set (if
  at all), and that member will be deleted.  
  \index{\bs{}in operator}
  \index{operator!\bs{}in}
\item \texttt{?Set[member->?Element]@\bs{}set} --- true if
  \texttt{?Element} is bound to a term that unifies with some member of the
  set identified by \texttt{?Set}.
  \\
  \ERGO provides a convenient shortcut for this method: \texttt{?Element
    \bs{}in ?Set}.  
\end{itemize}
%% 

\subsubsection{API for Working with Maps Only} \label{sec-api-map}

%% 
\begin{itemize}
\item  \texttt{?Map[mapinsert(?Key=?Value)]@\bs{}set} --- insert the
  corresponding key-value pair into the map. 

  If \texttt{?Key} already has a value then: if that value unifies with
  \texttt{?Value} then the above call succeeds; otherwise, it fails.
  For example, \texttt{map123[mapinsert(foo=23)]@\bs{}set}. 
\item \texttt{?Map[mapreplace(?Key=?Value)]@\bs{}set} --- like
  \texttt{mapinsert} but if the key has a value already,
  that old value is removed and \texttt{?Value} is associated with
  \texttt{?Key} instead. This call always succeeds.
\item \texttt{?Map[mapdelete(?Key)]@\bs{}set} --- delete the value
  associated with \texttt{?Key}. If a key-pair with that key exists,
  the call fails.
\item \texttt{?Map[maperase(?Key)]@\bs{}set} --- like \texttt{mapdelete}, but
  this call succeeds in all cases, whether the key in question exists or
  not.
\item \texttt{?Map[mapfind(?Key)->?Value]@\bs{}set} --- \texttt{?Value}
  gets bound to the value associated with \texttt{?Key} in \texttt{?Map}.
  Fails if no such key exists in the map.
\end{itemize}
%% 

\subsubsection{API for Working with MV-Maps Only} \label{sec-api-mvmap}

%% 
\begin{itemize}
\item  \texttt{?MV[mvminsert(?Key=?Value)]@\bs{}set} --- insert the
  corresponding key-value pair into the mv-map. 

  If \texttt{?Key} already has a value then: if that value unifies with
  \texttt{?Value} then the above call succeeds; otherwise, it fails.
  For example, \texttt{mv123[mvminsert(foo=23)]@\bs{}set}. 
\item \texttt{?MV[mvmdelete(?Key=?Value)]@\bs{}set} --- delete a
  \texttt{?Value} 
  associated with \texttt{?Key}. If a key-pair with that key exists,
  does nothing and succeeds.
\item \texttt{?MV[mvmerase(?Key)]@\bs{}set} --- like \texttt{mvmdelete},  but
  deletes all values associated with \texttt{?Key}. Succeeds regardless
  of whether \texttt{?Key} exists in the mv-map or not.
\item \texttt{?MV[mvmfind(?Key)->?Value]@\bs{}set} --- \texttt{?Value}
  gets bound to a value associated with \texttt{?Key} in \texttt{?MV}.
  Fails if no such key exists in the mv-map.
\end{itemize}
%% 





%%% Local Variables: 
%%% mode: latex
%%% TeX-master: "../ergo-manual"
%%% End: 
