
\section{Alerts} \label{sec-alerts}

\index{alert}
%%
An \emph{alert} statement is a request to monitor a predicate or a frame
(including isa- and subclass-literals) for certain predefined conditions.
Note: the monitored predicate cannot be a complex subgoal, nor can it be a
conjunction or disjunction, not even a \texttt{\bs{}naf} of something.
It \emph{can} be a \texttt{\bs{}neg} of a predicate or a frame, however. 
If the status of a predicate (or frame) monitored by an alert changes
as a result of an update transaction, the alert is raised.
At present, \ERGO supports these types of alerts:
%% 
\begin{itemize}
\item  \texttt{conflict} alerts
\item  \texttt{truth}  alerts
\end{itemize}
%% 
\index{alert!of type conflict}
\index{alert!of type truth}
A \texttt{conflict} alert is raised if a monitored predicate or frame
has a conflict \emph{after} a transaction and that conflict did not exist
\emph{before} the transaction.

For a true literal, \texttt{foo},  defined via \emph{non-defeasible} rules (see
Section~\ref{sec-defeasible}), this means
that \texttt{\bs{}neg} \texttt{foo} can also be derived and therefore the
information about \texttt{foo} is contradictory.
If foo is \emph{defeasible}, a conflict
means that there is an \textit{opposing}
literal \texttt{bar} (i.e., \texttt{\bs{}opposes}(\texttt{foo},\texttt{bar}) is
true) and both \texttt{foo} and \texttt{bar} are derivable if defeasibility is
disregarded.  Note that both \texttt{foo} and \texttt{bar} could be false due
to being defeated and yet they maybe in conflict (and this is likely the
reason why they were defeated).

\texttt{Truth} alerts are simpler. A truth alert for a literal \texttt{foo}
is raised after a transaction
if \texttt{foo} was false before  the transaction and became true after.

\index{alert!of type conflict}
\paragraph{Conflict alerts.}
To illustrate the concept of a conflict, suppose
\texttt{employee(Acme,John,2016)} means that John works for Acme and
\texttt{\bs{}neg employee(Acme,Bob,2017)} means that Bob was fired by Acme.
If neither \texttt{employee(foo,bar)} nor \texttt{\bs{}neg
  employee(foo,bar)} is true, it means that \texttt{foo} neither hired nor
fired \texttt{bar}.   The management might want to be alerted if the same
employee was hired and fired in the same year. Note that
this is not a constraint, as hiring and firing of the same employee in the
same year is \emph{not} an integrity violation.

\index{alert\{...\} primitive}
\index{primitive!alert\{...\}}
To tell \ERGO that we would like to be notified about the aforesaid events,
we can use the alert primitive:
%% 
\begin{verbatim}
    ?-  +alert{conflict,employee(?,?,?)}.
\end{verbatim}
%% 
Then, if \texttt{employee(Acme,John,2016)} already exists and we execute
the transaction \texttt{insert\{\bs{}neg employee(Acme,John,2016)\}} then
this alert will be raised:
%% 
\begin{verbatim}
?-  insert{\neg employee(Acme,John,2016)}.

*** Conflict alerts raised after transaction insert{\neg employee(Acme,John,2016)}:
        employee(Acme,John,2016)
    These alerts were activated on line 3 in file mykb.ergo
\end{verbatim}
%% 

\index{callback!in alert}
\index{alert!with callback}
\paragraph{Alerts with callbacks.}
Similarly to constraints, one can tell \ERGO to call a HiLog predicate when
an alert is raised. Again as in the case of a constraint, the callback
predicate must be at least one argument, and that argument must be unbound.
It will be bound by the system to the list of alerts discovered during the
transaction. 
For instance, we could define
%% 
\begin{verbatim}
    employee_alert_callback(?Alerts,?Module) :-
            insert{employee_alerts(?Alerts)},
            fmt_write('Employee alerts in module %s\n', arg(?Module))@\io,
            fmt_write('The alerts %S are logged.\n', arg(?Alerts))@\io.
\end{verbatim}
%% 
Then we could activate the alert using the 3-argument version of the
\texttt{+alert\{...\}} primitive:
%% 
\begin{verbatim}
    ?-  +alert{conflict,employee(?,?,?), employee_alert_callback(?,\@)}.
\end{verbatim}
%% 
Recall that \texttt{\bs{}@} above is a quasi-constant denoting the current
module. 

\paragraph{Complex alerts.} Lest the reader thinks that only simple
alerts can be specified using the above mechanism, here is a much more
involved example. Suppose we want to be notified if an employee 
was fired not only in the same year, but within less that 4 years. To do so, we
cannot simply use the \texttt{employee} 
predicate as before, but we can define what
we need via rules:
%% 
\begin{verbatim}
    longtime_or_current_employee(?Company,?Empl) :- employee(?Company,?Empl,?).
    \neg longtime_or_current_employee(?Company,?Empl) :-
            employee(?Company,?Empl,?Year),
            \neg employee(?Company,?Empl,?Year2),
            ?Year2 >= ?Year,
            ?Year2 - ?Year < 4.
    ?-  +alert{conflict,longtime_or_current_employee(?,?)}.
\end{verbatim}
%% 
Now, a conflict alert will be raised whenever an employee is fired within
4 years from hiring.  With a little bit of thinking, arbitrarily complex
alerts can be specified with the help of rules using the above idea.

\index{alert!of type truth}
\paragraph{Truth alerts.}
Truth alerts are defined similarly to conflict alerts except that
\texttt{truth} is used as the first argument instead of \texttt{conflict}:
%% 
\begin{verbatim}
    +alert{truth,longtime_or_current_employee(?,?)}.
\end{verbatim}
%% 
If some instances of \texttt{longtime\_or\_current\_employee(?Company,?Empl)} 
become true as a result of a transaction, a warning similar to conflict
updates will appear. For instance,
%%
\begin{verbatim}
*** Truth alerts raised after transaction insert{\neg employee(Acme,John,2016)}:
        longtime_or_current_employee(Acme,John)
        longtime_or_current_employee(GroupLTD,Bob)
    These alerts were activated on line 3 in file mykb.ergo
\end{verbatim}
%% 
Callbacks and deactivation of truth alerts are specified
analogously to conflict alerts.

Note that alerts of type \texttt{truth} have many similarities with
integrity constraints described in
Section~\ref{sec-kb-constraints}. Indeed, instead of
%% 
\begin{verbatim}
   ?-  +constraint{COI(?,?)}.
\end{verbatim}
%% 
in the examples in that section we could as well request
%% 
\begin{verbatim}
   ?-  +alert{truth,COI(?,?)}.
\end{verbatim}
%% 
and the results would be similar. Indeed, in both cases, the true instances of
the query \texttt{COI(?Company, ?Person)} would be reported after each
transaction. There are, however, several significant differences: 
%% 
\begin{itemize}
\item  An alert is \emph{not} conceptualized as an integrity constraint, so
  transactional updates will \emph{not} be rolled back, if the monitored
  predicate becomes true.
\item  An alert is raised only for the instances of the monitored
  predicate that became true as a result of the preceding transaction.
  Instances that were true before the transaction are not reported.
  In contrast, constraint violations are always reported---new and
  old---after each update transaction.
\item Alerts are more expensive computationally than constraints because
  they monitor the state of the knowledge base before and after each
  transaction. A constraint checks only the after-states.
\end{itemize}
%% 

Finally, we would like to remind something that important even if obvious:
each alert results in a query (to find prior conflicts or truths)
prior to evaluation of \emph{any}
top-level query. It also results in an execution of \emph{another} query
right after the previous top-level query, if the latter was a
state-changing transaction. Therefore, one should keep the number of active
alerts to absolute necessity and one should strive
to keep the monitored predicates rather lightweight.



%%% Local Variables: 
%%% mode: latex
%%% TeX-master: "../flora2-manual"
%%% End: 
