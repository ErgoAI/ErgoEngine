
  \subsection{Runtime Inspection of Computation}\label{sec-runtime}

  \index{showgoals\{...\} primitive}
  \ERGO allows the user to stop any computation by hitting \texttt{Ctrl-C}
  in the terminal or by clicking the \texttt{Pause} button in the \ERGO
  listener.
  During the pause, one can ask certain queries about the state of the
  computation, but not arbitrary queries. Specifically, no tabled
  (non-transactional) queries are allowed and insert/delete operations are
  not permitted either.
  The most useful queries that one \emph{can} ask are described below.

  The \texttt{showgoals\{...\}} command to display information about 
  the goals still being computed. The information shown is the number of
  answers for each goal computed so far and the number of calls to each goal.
  The output is sorted first by the number of answers and then by the
  number of calls. There are two types of the \texttt{showgoals} command:
  %% 
  \begin{itemize}
  \item    \texttt{showgoals\{\}}: this shows all incomplete subgoals that
    have over 1000 calls or over 50 answers.
  \item   \texttt{showgoals\{\normalfont{\emph{CallCutoff,AnwerCutoff}}\}}:
    This will show only the subgoals that have more than \emph{CallCutoff} calls or
    over \emph{AnswerCutoff} answers.  
  \end{itemize}
  %% 

  \index{showtables\{...\} primitive}
  \ERGO also provides a companion primitive,  \texttt{showtables\{...\}},
  which is usually more convenient
  than the table dump because it provides a more focused output and is
  easier to use. This primitive is intended not for the pause in computation
  but for analysis to be performed
  after a query of interest is finished. It then displays subgoals
  that took part in the computation of the last (and previous queries)
  using the same format as \texttt{showgoals\{...\}}.
  There are two types of the \texttt{showtables} command:
  %% 
  \begin{itemize}
  \item    \texttt{showtables\{\}}: this shows all incomplete subgoals that
    have over 1000 calls or over 50 answers.
  \item   \texttt{showtables\{\normalfont{\emph{CallCutoff,AnwerCutoff}}\}}:
    This will show only the subgoals that have over \emph{CallCutoff} calls or
    more than \emph{AnswerCutoff} answers.  
  \end{itemize}
  %% 

  The \texttt{showgoals\{...\}} primitive is typically used to investigate
  reasons for \emph{runaway} computation, i.e., non-termination of queries
  as well as diagnose performance problems. It is designed to be used
  during pauses in query computation, before the query finishes. The
  \texttt{showtables\{...\}} primitive is designed to be used after the
  query finishes and thus is useful only for analyzing the performance
  aspect of runaway computations. The call and answer cutoffs must be
  chosen judiciously.
  In both of the
  above runaway cases, the goals of interest are those that either have
  many calls to them (in the thousands, but depends on the case at hand) or
  have many answers (in the tens or hundreds, as appropriate).  Examining
  the information provided by \ERGO during pauses and after query
  completion can provide a number of insights:
  %% 
  \begin{itemize}
  \item \emph{Compare the number of active goals and the number of active
      recursive
      components,} which is shown when the computation is paused.
    The active goals/components are the ones that are currently being computed
    and therefore are of interest during pauses in the computation.
    %% 
    \begin{itemize}
    \item Check the \emph{ratio} of the \emph{number of active recursive
        components to the number of active goals.}  This ration is
      normally $\leq 1$, but if it is very small (say, under 3\% and keeps
      falling during subsequent pauses), then it is an indication that
      the rules are poorly structured. Every subgoal depends on many other
      subgoals and there likely are too many calls that do not contribute
      to the query results, but just waste the computational
      resources. Consider restructuring the rules so that the recursive
      components form some kind of a hierarchy. The use of \ERGO modules is
      strongly recommended in this case. Also, try to mix HiLog and
      frame-based representation.
  \item If the aforesaid ratio stays
    approximately constant and \emph{both} the number of active recursive
    components and the number of active goals climb steadily,
    it is an indication of an infinite recursion where subgoals are
    called with increasingly deeply nested terms (e.g., \texttt{p(f(a))},
    \texttt{p(f(f(a)))}, \texttt{p(f(f(f(a))))}, etc.).
    The aforesaid ratio in such cases is typically non-negligible ($> 0.3$).
    \end{itemize}
    %% 
  \item \emph{Check the memory usage}, which is shown during each pause.
    If memory usage keeps climbing steadily, it might indicate a runaway
    computation or a possible scalability problem.
  \item \emph{Watch the cpu time} shown during each pause.  It can be
    helpful as a way of tracking progress.
    This is cpu time since the time \ERGO started; it includes all the
    pauses.
  \item \emph{Check the number of derived facts} shown at each pause.
    If this number does not grow and \ERGO keeps going on and on, it is an
    indication of some kind of a runaway computation.
  \item \emph{Use \texttt{showgoals} during the computation pauses.}
    As mentioned earlier, you will likely be interested in goals with many
    calls or many answers. Keep increasing the cutoffs during subsequent
    pauses. Examining the active goals can indicate two things:
    %% 
    \begin{itemize}
    \item   Weird, unexpected stuff showing up as active goals
      might provide a hint about a possible bug in the user's rules.
    \item  Some goals have unusually high numbers or calls and/or answers.
      This may mean two things:
      %% 
      \begin{itemize}
      \item      Non-terminating computation. The goals in question might
        be involved in a non-terminating recursive loop. \emph{Terminyzer}
        (Section~\ref{sec-terminyzer}) may
        supply additional information if this is the case, including the
        information about the rules that are involved in such a loop.
      \item  Gross inefficiency. The computation may be terminating but,
        for some reason, the goals in question get called unusually often or
        they produce too many results. The latter might also indicate a
        bug.
      \end{itemize}
      %% 
    \end{itemize}
    %% 

  \item \emph{Use \texttt{showtables} after the computation.}
    This is used after and if the query finishes. In that case, the query
    obviously terminates, but one might be still dissatisfied with the
    performance. The \texttt{showtables} primitive supplies the same kind
    of information as \texttt{showgoals}, but since the former is called
    after the query is done, it supplies complete information about the
    subgoals that took part in the computation.
    This can give too kinds of hints:
    a possible bug or a possible inefficiency. As before, a bug may be
    lurking if unexpected subgoals show up in the output of
    \texttt{showtables}. Inefficiency should be suspected if some subgoals
    have unusually high number of calls and/or answers.

    The table dump described in Section~\ref{sec-tabledump} provides much
    additional information, but examining it manually is usually overwhelming.
    The table dump feature is designed for expert users who know how to
    write Prolog and/or \ERGO programs to analyze such dumps automatically
    and check the various hypotheses about the computational behavior of
    queries.
  \end{itemize}
  %% 

  Another very useful informational primitive that is allowed during pauses is
  \texttt{peek\{}\emph{query}\texttt{\}}, where \emph{query} is an atomic
  subgoal (tabled, non-transactional). It examines the current state of
  the computation of \emph{query} and reports the answers that have 
  been computed so far. This primitive is described in more detail in
  Section~\ref{sec-peek}.

\subsection{Continuous Runtime Monitoring}\label{sec-monitor}

Sometimes, especially when queries take a long time, it might be useful to see
the various statistics pertaining the queries. To request this service, the
user can execute the following command in the \FLSYSTEM shell:
\index{setmonitor{...}}
%% 
\begin{alltt}
   \prompt setmonitor\{\emph{Secs},\emph{Type}\}. 
\end{alltt}
%% 
The effect will be that all subsequent queries will be monitored and every
\emph{Secs} seconds certain statistics will be printed out to the standard
output. Here \textit{Type} is the type of
the monitor. The \texttt{heartbeat} monitor just shows the elapsed time.
The \texttt{performance} monitor shows time, memory, and other
key statistics. The \texttt{extended} monitor shows additional
statistics of interest.

\emph{Secs} must be a non-negative integer. If \emph{Secs} = 0, monitoring
is turned off.  Note: it is not guaranteed that the statistics will be
refreshed every \emph{Secs} seconds, as certain intermediate
computations are not
interruptable and may take several seconds. In such cases, the next
statistics report would be output right after such a non-interruptable
computation ends.

If the user instead types
%% 
\begin{alltt}
   \prompt setmonitor\{\}. 
\end{alltt}
%% 
then the parameters can be input via a dialog window, which will pop up.

If \ERGO is run under the studio or in a graphical desktop environment out
of a command line window, the output from the monitor will go into a
separate pop-up window. Otherwise, it will be redirected to the standard
output (which might
interfere with the output of the query \emph{if}  the query has
write-statements of its own; this \emph{does not} interfere with the
display of the query results however).

Some of the statistics shown by the \texttt{performance} and
\texttt{heartbeat} monitors are described in Section~\ref{sec-runtime} on
runtime inspection of computation and they thus provide indication
of the state of the system as described there.

%%% Local Variables: 
%%% mode: latex
%%% TeX-master: "../ergo-manual"
%%% End: 
