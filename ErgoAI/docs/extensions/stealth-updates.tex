
\subsection{Stealth Updates}\label{sec-stealth-updates}

Issuing an update command while still evaluating a non-transactional
subgoal in a reactive tabling module will cause a runtime error:
%% 
\begin{verbatim}
   p(?X) :- insert{%abc}.
   ?- p(1).

   ++Error[Ergo]> tabled predicate/method depends on an update.
\end{verbatim}
%% 
This happens because, while computing the truth value of \texttt{p(1)}, we
changed the underlying knowledge base, which could have affected the result.
In this particular case, inserting \texttt{\%abc} cannot affect \texttt{p(1)}
and the inference engine could have noticed this. However, maintaining
precise dependencies among the literals is expensive and is usually not
worth the effort and the computational cost. In general, of course, inserting \texttt{\%abc} could have affected
the truth value of \texttt{p(1)}, which makes \texttt{p(1)} even harder to
evaluate, whence the error.

As discussed earlier, the above type of dependency is bad style and
issuing an error makes perfect sense. However, in some situations, like
the example of computing histograms on page~\pageref{page-histogram},
updates issued while evaluating tabled subgoals can be useful.
Another situation when this facility could be used is when one needs to do
some kind of bookkeeping such as recording if a particular tabled literal
has ever been called.

\index{update!stealth}
\index{stealth update}
\index{stealth\{...\}}
%%
To this end, \ERGO provides the \texttt{stealth\{...\}} primitive, which
can be used in insert and delete commands to make literals being
inserted/deleted invisible to the reactive tabling mechanism, thus
avoiding runtime errors. For instance,
%% 
\begin{verbatim}
   p(?_X) :- insert{stealth{%abc,%p(2),a[%b]}}.
   ?- p(1).

   Yes
\end{verbatim}
%% 
And yes, only transactional literals, as in the above example, can be
inserted/deleted with the help of the \texttt{stealth\{...\}} primitive. 
Without this primitive, even transactional literals cannot be used in updates
while evaluating a tabled subgoal in a reactive module.

Note that transactional variables \emph{are} allowed in stealth updates. For example,
%% 
\begin{verbatim}
   ergo> ?X = ppp, insert{stealth{%?X(1)}}.
   ?X = ppp

   ergo> %ppp(?X).
   ?X = 1
\end{verbatim}
%%

\index{UDF!and stealth update}
\index{stealth update and UDFs}
Another interesting use of stealth updates is defining UDFs (see
Section~\ref{sec:udf}) that have side
effects like database updates. For instance, suppose we want a UDF to
insert something into the database and return some result:
%% 
\begin{verbatim}
   %fooPRED(?val, =?Res+1) :- insert{%abc(?val)}, ?Res = ?val.
   \udf fooUDF(?form) := ?res \if %fooPRED(?form, ?res).
\end{verbatim}
%% 
Unfortunately, this won't work because if one asks the query \texttt{?-
?A=fooUDF(1)} then an error will result for the same reason as earlier in
this section. However, changing \texttt{insert\{\%abc(?val)\}}  to
\texttt{insert\{stealth\{\%abc(?val)\}\}} saves the day. 



%%% Local Variables: 
%%% mode: latex
%%% TeX-master: "../flora2-manual"
%%% End: 
