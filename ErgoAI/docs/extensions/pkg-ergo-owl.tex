\chapter[Loading RDF and OWL files]
{Importing RDF and OWL\\
  {\Large by Paul Fodor and Michael Kifer}}

This chapter describes the \ERGO import facility for RDF and OWL files.
The Resource Description Framework (RDF) and the Web Ontology
Language (OWL) are families of knowledge representation languages for
authoring ontologies.

The \ERGO-to-OWL API is available through the \ERGO system module
\texttt{\bs{}owl}
and calling anything \texttt{@\bs{}owl}  will load that module. If, however, for
some reason it is necessary to load this module without executing any
operations, one can accomplish this by calling 
%% 
\begin{itemize}
\item  \texttt{ensure\_loaded@\bs{}owl}. 
\end{itemize}
%% 

\section{Loading RDF and OWL Files}

The main predicate for importing and loading RDF and OWL files into \ERGO is
\texttt{rdf\_load}:

  %%\texttt{System[rdf\_load(?InputFileName, ?InputLangSyntax,
    %%\\
    %%\hspace*{3cm} ?LoadingMethod, ?IriPrefixes, ?RdfModule)]@\bs{}owl}.
  \texttt{System[rdf\_load(?InputFileName,?InputLangSyntax,?IriPrefixes,?RdfModule)]@\bs{}owl}.

  The parameters of this query are explained below. They are all input
  parameters and therefore must be bound. The result of the translation is
  stored in an \ERGO module indicated by the last argument.
  
  \texttt{?InputFileName}  must be bound to an \ERGO symbol (Prolog atom);
  it is an input file name where the RDF or OWL file resides 
  (this can be absolute or relative path).
  It is advisable that the user uses forward slash as the delimiter for
  specifying path names.
  Backslash also works, but it should be doubled, as backslashes
  need to be escaped.

  Note: the input file name can be a URL in which case it should have the
  form \texttt{url(\textnormal{\emph{the-web-address}})}. For example,
  \texttt{url('http://www.w3.org/TR/owl-guide/wine.rdf')}. 
  This feature will work only if the host system has all the required
  software installed (like the \emph{curl} package. (Please refer to the
  installation instructions.) 
  

  \texttt{?InputLangSyntax}  must be bound to an \ERGO symbol (Prolog atom);
  it is an input file syntax: 
  'RDF/XML',
  'JSON-LD'
  'TURTLE',
  'TTL'	,
  'N-TRIPLES',
  'N-QUADS',
  'NT',
  'N3', or
  'RDF/JSON' (lowercase versions are also accepted).

  If \texttt{?InputLangSyntax} is an empty atom \texttt{''} then the input
  syntax is determined from the file extension.\footnote{
    \texttt{.owl} and \texttt{.rdf}  for  'RDF/XML',
    \texttt{.nt} for 'N-TRIPLES' and 'NT',
    \texttt{.ttl} for 'TTL' and 'TURTLE',
    \texttt{.nq} for 'N-QUADS', 
    \texttt{.jsonld} for 'JSON-LD',
    \texttt{.rj} for 'RDF/JSON',
    \texttt{.n3} for 'N3'. 
    }

%%  The flag \texttt{?LoadingMethod} indicates the method to use for RDF/OWL
%%  import. \ERGO translates the original RDF/OWL files into an intermediate
%%  format and then loads that intermediate file.\footnote{
%%    These intermediate files are created in a hidden directory
%%    \texttt{\emph{USERHOME}/.xsb/ergo-\emph{VERSION}/ergo\_owl/},
%%    but the user normally does not need to see them.
%%  }
%%  The intermediate format depend on the specified method.
%%  The \texttt{?LoadingMethod} must be bound to one of these symbols:
%%  %% 
%%  \begin{itemize}
%%  \item   \texttt{fastload}\\
%%    This method represents RDF triples via an obscure, but efficient
%%    format. The triples are loaded as facts into the module specified by the
%%    argument \texttt{?RdfModule}.
%%    They can be queries via the \ERGO frame syntax
%%    \texttt{?S[?P->?O]@rdfStorage123} (assuming \texttt{?RdfModule} is
%%    bound to \texttt{rdfStorage123}). 
%%    \\
%%    \texttt{fastload} is the \emph{preferred} method, as it is much more
%%    efficient than the other two methods
%%    for large files. The intermediate file created
%%    by this method
%%    is a bit hard for humans to read, so it is not appropriate
%%    for debugging. Nevertheless, users normally do not need to inspect the
%%    intermediate files into which \ERGO translates RDF and OWL, so the
%%    readability of this format is rarely an issue.
%%    \\
%%    Example of a query:\\
%%    \texttt{System[rdf\_load('wine.owl','',fastload,'',rdfStorage123)]@\bs{}owl,\\?S[?P->?O]@rdfStorage123.}  
%%  \item \texttt{predicates}\\
%%    In this method, RDF triples are represented as
%%    \texttt{property(subject,object)}, and they are loaded into
%%    the \ERGO module specified by the
%%    \texttt{?RdfModule} argument. They are queried as
%%    \texttt{?P(?S,?O)@rdfStorage123}, where we assume that the argument
%%    \texttt{?RdfModule} is bound to \texttt{rdfStorage123}.
%%    \\
%%    This format uses the regular \ERGO syntax for the intermediate files it
%%    creates, and so it is easier to inspect
%%    visually than the intermediate format for the \texttt{fastload} method.
%%    However, the need to inspect these intermediate files is very rare,
%%    while loading large files in this format is much slower than for
%%    \texttt{fastload}. \\
%%    Example of a query:\\
%%    \texttt{System[rdf\_load('wine.owl','',predicates,'',rdfStorage123)]@\bs{}owl,\\?P(?S,?O)@rdfStorage123.}  
%%  \item \texttt{frames}\\
%%    In this method, RDF triples are represented using the \ERGO frame
%%    syntax:
%%    \texttt{subject[property->object]}.
%%    These facts are also loaded into the
%%    \ERGO module specified by \texttt{?RdfModule}. They are queried
%%    similarly to the \texttt{fastload} method: 
%%    \texttt{?S[?P->?O]@rdfStorage123}, where
%%    \texttt{rdfStorage123} is the binding for \texttt{?RdfModule}
%%    in this example.
%%    \\
%%    As with the \texttt{predicates} method, loading large RDF/OWL files via
%%    the \texttt{frames} method is much slower than via the
%%    \texttt{fastload} method.   
%%    \\
%%    Example of a query:\\
%%    \texttt{System[rdf\_load('wine.owl','',frames,'',rdfStorage123)]@\bs{}owl,\\?S[?P->?O]@rdfStorage123.}  
%%  \end{itemize}
%%  %% 
%%  

  \texttt{?IriPrefixes} must be bound to an \ERGO symbol (Prolog atom) and
  be a sequence or rows, ending with the newline character,
  where each row has the form \texttt{prefix=URL}: 
  %% 
\begin{verbatim}
     'prefix1=URL
      prefix2=URL2
      ...
      prefixN-URL_N'
\end{verbatim}
  %% 
  This parameter can be used to define prefixes for compact URIs (curi's)
  that are
  used inside the input RDF/OWL files but not defined in that file. These prefixes will be added to the
  standard pre-defined prefixes \texttt{rdf}
  (\url{http://www.w3.org/1999/02/22-rdf-syntax-ns#}), \texttt{rdfs}
  (\url{http://www.w3.org/2000/01/rdf-schema#}), \texttt{owl}
  (\url{http://www.w3.org/2002/07/owl#}), and
  \texttt{xsd} (\url{http://www.w3.org/2001/XMLSchema#}).
  If any of the standard prefixes \texttt{rdf}, \texttt{rdfs},
  \texttt{owl}, or \texttt{xsd}     are also defined in
  \texttt{?IriPrefixes}, the latter override the default definitions.

  Note: this does \emph{not} define any IRI prefixes in \ERGO itself.
  If one wants to use the above prefixes inside \ERGO, the following
  queries must be issued:
  %% 
\begin{verbatim}
 ?- iriprefix{prefix1=URL},
    iriprefix{prefix2=URL2},
    ....
\end{verbatim}
  %% 
  Note that URL, URL2, etc., must be symbols and thus be
  quoted correctly, as in
  %% 
\begin{verbatim}
  ?- iriprefix{foo='http://my.foo.bar.com/foobar'}.  
\end{verbatim}
  %% 

  The last argument,
  \texttt{?RdfModule}, must be bound to an \ERGO symbol (Prolog atom);
  it indicates the \ERGO module into which the RDF imported triples
  should placed at run time. These triples have the form
  \texttt{?Subject[?Property->?Object]} and can be queried as follows:
  %% 
  \begin{quote}
    \texttt{?Subject[?Property->?Object]@MyRdfModule}  
  \end{quote}
  %% 
  where we assume that \texttt{?RdfModule} is bound to
  \texttt{MyRdfModule} in this example. 
  

  \paragraph{A simplified version of the rdf\_load query.}
  In most cases the user does not need to use all the options
  provided by the \texttt{rdf\_load} method and the following query would
  suffice: 

  \texttt{System[rdf\_load(?InputFileName, ?RdfModule)]@\bs{}owl}

  The input language
  syntax is determined from the file extension and no IRI prefixes are
  expected to be supplied. In other words, a call like
  
  \texttt{System[rdf\_load('wine.owl', MyRdfModule)]@\bs{}owl}

  is equivalent to

  \texttt{System[rdf\_load('wine.owl','','',MyRdfModule)]@\bs{}owl}

\section{Other API Calls}

Besides loading, the following API calls are supported:

%% 
\begin{itemize}
\item  \texttt{?RdfModule[rdf\_insert(?S,?P,?O)]@\bs{}owl} --- insert
  \texttt{?S[?P->?O]} into the RDF module indicated by \texttt{?RdfModule}.
\item  \texttt{?RdfModule[rdf\_delete(?S,?P,?O)]@\bs{}owl} --- delete
  a fact that matches \texttt{?S[?P->?O]} from the RDF module indicated by
  \texttt{?RdfModule}.
\item  \texttt{?RdfModule[rdf\_deleteall]@\bs{}owl} --- empty out the
  specified RDF module.
\item \texttt{?Subject[rdf\_reachable(?RdfModule,?Property)->?Object]@\bs{}owl}
  --- true if \texttt{?Object}  is reachable from \texttt{?Subject} via a
  path of properties \texttt{?Property}. If the property is specified
  simply as \texttt{?} then any path will do. If \texttt{?Property} is
  bound (say, to \texttt{foo})  then only the paths consisting of the
  \texttt{foo}-edges will be considered. If \texttt{?Property} is an
  unbound variable then any path will do, if all edges of that path are the
  same. 

\item \texttt{?RdfModule[rdf\_predicate->?P]@\bs{}owl},
  \texttt{?RdfModule[rdf\_subject->?S]@\bs{}owl},
  \texttt{?RdfModule[rdf\_object->?O]@\bs{}owl} --- return the set of all
  properties, subjects, and objects, respectively, in the RDF module
  \texttt{?RdfModule}. 
\end{itemize}
%% 

\section{Importing Multiple RDF/OWL Files}

Multiple RDF and OWL files can be loaded into separate \ERGO modules (and
the same file can even be loaded into different modules, if so desired).
However, what happens if two files are loaded into the \emph{same} module? For instance,
%% 
\begin{verbatim}
  ?- System[rdf_load('wine.owl', MyRdfModule)]@\owl,
     System[rdf_load('beer.owl', MyRdfModule)]@\owl,
     ... ... ... do something ... ... ...
\end{verbatim}
%% 
In that case, the data from the second import will be \emph{added} to the data
obtained from the second import. If this additive behavior is not what is
required in a particular situation and one wants the second import to
\emph{override} the first, a call to \texttt{rdf\_deleteall} will
do the trick:
\begin{verbatim}
  ?- System[rdf_load('wine.owl', MyRdfModule)]@\owl},
     ... ... ... do something ... ... ...
     MyRdfModule[rdf_deleteall]@\owl,  // erase the previously imported data
     System[rdf_load('beer.owl', MyRdfModule)]@\owl},
     ... ... ... do something else ... ... ...
\end{verbatim}




%%% Local Variables: 
%%% mode: latex
%%% TeX-master: "../flora-packages"
%%% End: 
