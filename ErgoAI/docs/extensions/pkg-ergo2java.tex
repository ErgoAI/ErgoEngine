\chapter[\ERGO-to-Java Interface]
{\ERGO-to-Java Interface: Calling Java from \ERGO\\
  {\Large by Michael Kifer}}


This chapter describes the API for opening some standard Java widgets from
within \ERGO rules. This API also allows one to call arbitrary Java
programs and thereby use \ERGO for scripting Java applications.

The \ERGO-to-Java API works both when \ERGO runs as a standalone
application and when it is under the control of Ergo Studio.
The API calls should work the same in either environment.

\section{General}

The \ERGO-to-Java API is available in the system module \texttt{\bs{}e2j}
and calling anything in this module will load that module. If, however, for
some reason it is necessary to load this module without executing any
operations, one can accomplish this by calling 
%% 
\begin{itemize}
\item  \texttt{ensure\_loaded@\bs{}e2j}. 
\end{itemize}
%% 

The following additional general API calls are available:
%% 
\begin{itemize}
\item  \texttt{System[mode->?Mode]@\bs{}e2j} - the variable \texttt{?Mode} will be bound to one
  of the following:
  %% 
  \begin{itemize}
  \item    \texttt{studio} -- if \ERGO runs as part of Ergo Studio.
  \item    \texttt{[ergo2java,gui]} -- if \ERGO runs as a standalone mode
    in an environment that supports graphics. This is usually the case when
    one invokes \ERGO in a command window on a personal computer.
  \item \texttt{[ergo2java,nogui]} -- this is usually the case when \ERGO
    runs in a non-graphical environment, such as a dumb terminal or a
    command window opened on a remote server.
    In a \texttt{nogui} situation, none of the widgets (windows, dialogs,
    etc.) will be available. However, the dialog boxes will be simulated
    through a command-line interface.
  \end{itemize}
  %% 
  \item \texttt{System[restart]@\bs{}e2j} -- restarts the Java subprocess, if it was
    killed and is needed again. This is required very rarely: for instance,
    when the Java subprocess was killed outside of \ERGO (e.g., via the Task
    Manager or System Monitor). Java is also killed when
    \texttt{\bs{}end} is executed at the \ERGO prompt.
  \item \texttt{System[path(studioLogFile)->?File]@\bs{}e2j} -- also a rarely used
    feature. The variable \texttt{?File} gets
    bound to the location of the Studio log file. This calls fails outside
    of the studio environment. In the future, this API call will be extended
    to include other file locations that might be deemed useful in the
    future.
\end{itemize}
%% 

\section{Dialog Boxes}

This part of the API allows the user to pop up various dialog boxes and
find out which button was clicked by the user. Several types of dialog
boxes are supported:
%% 
\begin{itemize}
\item \texttt{Dialog[show(?Question)->?Answer]@\bs{}e2j} -- pops up a dialog box
  that asks the user a question and provides
  an input text field plus the buttons \texttt{OK} and
  \texttt{Cancel}.   
  If the user clicks \texttt{Cancel} the call fails. Otherwise, if
  \texttt{OK} is clicked, \texttt{?Answer} gets bound to whatever the user
  typed in the input field.   
\item
  \texttt{Dialog[showOptions(?Title,?Message,?Buttons)->?ChosenButton]@\bs{}e2j} --
  opens up a dialog box where the user is presented with a number of
  buttons to click on. Here \texttt{?Title} must be bound to an atom---it
  will be the title of the window;  \texttt{>Message} is an atom that
  contains the message to be displayed to the user (e.g., ``Please click a
  suitable button''); and \texttt{?Buttons} is a list of labels to appear
  on the buttons presented as the available choices (e.g.,
  \texttt{[Milk,Bread,Honey]}).
\item \texttt{Dialog[show(?Title,?Message)]@\bs{}e2j} -- pops up a dialog box that 
  shows a message (\texttt{?Message}) and waits until the user clicks
  \texttt{OK}.  \texttt{?Title} is the title of the dialog box.
\item \texttt{Dialog[chooseFile->?File]@\bs{}e2j} -- pops up a file
  chooser. \texttt{?File} gets bound to the file chosen by the user.
\item \texttt{Dialog[chooseFile(?ExtensionsList)->?File]@\bs{}e2j} -- like the above, but
  also takes a parameter that represents a \emph{list} of  file
  extensions. Only the files
  with that extensions mentioned in the list are shown to the user in the
  file chooser.
\end{itemize}
%% 

\section{Windows}

This part of the API supports opening, closing, and other operations on windows.
%% 
\begin{itemize}
\item \texttt{Window[open(?WindTitle,?Tooltip)->?Window]@\bs{}e2j} -- pops up a new
  window with the title \texttt{?WindTitle} and the tooltip
  \texttt{?Tooltip}. The tooltip is appears when the mouse rests over the
  window. The variable \texttt{?Window} gets bound to the Id of the newly
  created window. This Id will need to be passed to other API calls that
  manipulate windows, so the user must usually store these Ids in some
  predicates.
\item \texttt{Window[setSize(?Win,?Columns,?Rows)]@\bs{}e2j} -- changes the size of
  the window so it will have the given number of columns and rows.
  The system will then try to adjust the window (whose Id is passed in the
  first argument \texttt{?Win}) to approximate the requested size.
\item \texttt{Window[close(?Window)]@\bs{}e2j} -- closes the specified window.
\item \texttt{Window[alive(?Window)]@\bs{}e2j} -- tells if the window is alive
  (i.e., not closed by the user---either programmatically or by clicking
  the \texttt{x} button in the corner of the window). 
\end{itemize}
%% 

\section{Printing to a Window}

The following describes how to print to a previously open window and how to
erase the window contents.
%% 
\begin{itemize}
\item  \texttt{Window[clear(?Window)]@\bs{}e2j} -- erases the contents of the given
  window. 

\item \texttt{Window[print(?Window,?Text)]@\bs{}e2j} -- prints \texttt{?Text} to a
  given window.  \texttt{?Text} specifies what to print and how. Several
  colors are supported (\texttt{black}, \texttt{red}, \texttt{brown},
  \texttt{green}, \texttt{purple}, \texttt{blue}, \texttt{magenta},
  \texttt{orange}, and \texttt{default}), as well as a few faces
  (\texttt{italic}, \texttt{bold}, \texttt{boldital}).

  \texttt{?Text} is either a \emph{text descriptor} or a \emph{list} of
  text descriptors, where a text descriptor is
  %% 
  \begin{itemize}
  \item  a Hilog term; or
  \item  \emph{modifier}(Hilog term)
  \end{itemize}
  %% 
  Here \emph{modifier} is one of the aforesaid colors or faces.
  Not all faces may be available for the default fonts on your system
  so, say, \texttt{boldital} may appear as \texttt{italic} ot as
  \texttt{bold}. Likewise, colors may look different on different screens.   

  Note that
  if you want to print a term like \texttt{red(tomato)} then you would have
  to wrap it in one of the above modifiers, like
  \texttt{default(red(tomato))} (to print \texttt{red(tomato)}  in the default
  color---usually black) or \texttt{green(red(tomato))} (to print
  {\color{green}\texttt{red(tomato)}}). Otherwise, if \texttt{red(tomato)}
  is not wrapped as described, 
  \texttt{{\color{red}tomato}}  will be printed instead.

  Examples. Let us assume that window with Id 3 is open. Then:\\
  \texttt{Window[print(3,magenta('this is red(herring), 1lb'))]@\bs{}e2j}
  will print
  {\color{magenta}this is red(herring), 1lb}.\\
   \texttt{Window[print(3,[magenta('this is a '), green(2),
       italic(' pound '), red(herring)])]@\bs{}e2j}
     will print: {\color{magenta}this is a} {\color{green}2} \emph{pound}
     {\color{red}herring}. 
\end{itemize}
%% 


\section{Scripting Java Applications}

The java scripting API allows the user to load Java jar-files, invoke
methods that exist in the public classes of those jar-files, and process
the results.

%% 
\begin{itemize}
  %% addJar is deprecated
  %% \item  \texttt{System[addJar(?Jar)]@\bs{}e2j} -- load the specified
  %%   jar-file into the system. This API call works with Java 8, but not Java 9+.
\item \texttt{System[setJavaCP(\textnormal{\emph{Classpath}})]@\bs{}e2j} --
  specify the class path for additional Jars and classes to be used at run
  time. \emph{Classpath} must be an atom that represents a valid
  specification for the Java \texttt{-classpath} option.  Note that on
  Windows the elements of a classpath are separated with a ``;'' and on
  Linux/Mac with a ``:''.  Typically this call is made before issuing any
  other calls to \texttt{\bs{}e2j}.  If this is called after other calls to
  \texttt{\bs{}e2j} were issued, the existing Java process is killed first
  and any in-memory data that it may have
  created (e.g., objects that were created) will be discarded.
\item
  \texttt{\textnormal{\emph{JavaObjSpec}}[message(\textnormal{\emph{JavaMethodWithArgs}})
    -> \textnormal{\emph{Result}}]@\bs{}e2j} -- invoke Java method on a
  Java object and return the result.
  \\
  This feature is \emph{experimental} and incomplete. It does not support
  all Java data structures and not all kinds of methods can be applied. 
\end{itemize}
%% 
\emph{JavaObjSpec} in the above \texttt{message(...)}   API can have several forms:
%% 
\begin{itemize}
\item  \texttt{oid(}\emph{Integer}\texttt{)}: When Java returns an object,
  it is registered by \FLSYSTEM and is represented by an integer, e.g.,
  345. In order to 
  invoke a Java method on it, the \emph{JavaObjSpec} must be specified as
  \texttt{oid(345)}.
\item If \emph{JavaObjSpec} is an integer, float, or atom in \ERGO then
  it is interpreted as a Java long, double, or string, respectively. Java
  methods that apply can be used in \texttt{JavaMethodWithArgs}. For instance,
  %% 
\begin{verbatim}
?- '123abc789'[message(split(abc))->?P]@\e2j.
?P = ['123', '789']
?- abc23op[message(matches('.+23.*'))->?P]@\e2j.
?P = \true
?- abcdf[message(getBytes)->?R]@\e2j.
?R = "abcdf"^^\charlist
\end{verbatim}
  %% 
\item If \emph{JavaObjSpec} is a list, it is interpreted as an array of
  objects. For instance, the list \texttt{[pp,i(8),k(u,m)]} is mapped into
  a Java array and \texttt{[1]} is a method applied to that array, which
  returns the second element.  
  %% 
\begin{verbatim}
?- [pp,i(8),k(u,m)][message([1])->?P]@\e2j.
?P = i(8)  
?- [pp,i(8),k(u,m)][message(length)->?P]@\e2j.
?P = 3
\end{verbatim}
  %% 
\item If \emph{JavaObjSpec} is \texttt{byte(}\emph{smallNumber}\texttt{)},
  \texttt{short(}\emph{shortinteger}\texttt{)},
  or \texttt{int(}\emph{integer}\texttt{)}
  then it is interpreted as a byte or int constant.
\item If \emph{JavaObjSpec} is \texttt{term(}\emph{someHiLogTerm}\texttt{)} 
  then \emph{someHiLogTerm} is mapped into a \texttt{TermModel} object---see
  \url{http://interprolog.com/ipjavadoc/com/declarativa/interprolog/TermModel.html}
  for the details of this class, which has methods to de/compose terms on
  the Java side.
  %% 
\begin{verbatim}
?- term(p(a,b))[message(getFunctorArity)->?P]@\e2j.
?P = 'p/2'
\end{verbatim}
  %% 
\item If \emph{JavaObjSpec} is of the form
  \texttt{array(}\emph{type},\emph{list}\texttt{)} then this is mapped to
  an array of constants of the given type (string, byte, int, float). For
  instance
  %% 
\begin{verbatim}
?- array(int,[2,7,99])[message([2])->?P]@\e2j.
?P = 99
\end{verbatim}
  %% 
\item If \emph{JavaObjSpec} is of the form \emph{Wrap(const)}, where
  \emph{Wrap} is one of Boolean, Character, Byte, Double, Float,
  Integer, Long, or Short and \emph{const} is of the appropriate \ERGO
  type, then this will be mapped into an object of type
  \texttt{java.lang.Boolean}, \texttt{java.lang.Character}, etc., respectively.
  Note that these are objects, while the wrappers \texttt{int},
  \texttt{short}, \texttt{byte}, etc., which we introduced earlier   
  are non-object constants.
\item Additionally, to be able to execute \emph{static methods}, fully-qualified
  classes wrapped with \texttt{class} can be used. For instance,
  %% 
\begin{verbatim}
?- class('java.lang.String')[message(format('abc=%d %s',12,iiii))->?R]@\e2j.
?R = 'abc=12 iiii'
?- class('java.lang.String')
              [message(String(array(byte,[119,111,114,108,100])))->?P]@\e2j.
?P = world
\end{verbatim}
  %% 
  \item It is further possible to send a message to a \emph{static variable}
    (i.e., invoke a method on the object held by that static variable, as
    in \texttt{java.lang.System.out.println("Hello")}) 
    in a class as follows:
    %% 
\begin{verbatim}
?- class('java.lang.System'+out)[message(println(Hello))->?R]@\e2j.
Hello
?R = \@?   // a null value because println returns void
?- class('java.lang.System'+out)[message(printf('%1.16g', 69.1))->?R]@\e2j.
69.10000000000000
?R = oid(1)    
\end{verbatim}
    %% 
    Note that here \texttt{out} is the static variable in class
    \texttt{java.lang.String} to which the messages are being sent.  

    Currently it is not possible to get the value of a static variable
    unless there is a getter-method for that variable.
\end{itemize}
%% 
The same mapping conventions are applied to the arguments of the
method-expressions passed to \texttt{message(...)} in our API call. 



%%% Local Variables: 
%%% mode: latex
%%% TeX-master: "../flora-packages"
%%% End: 
