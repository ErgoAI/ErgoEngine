
\subsection{Examining Query Results: The peek\{...\} primitive}
\label{sec-peek}

\index{peeking into query results}
%%
\ERGO also provides a way to browse the results of the previous tabled
(non-transactional) queries, called \emph{peeking}, and is provided via
a special primitive 
%%
\index{peek\{...\} primitive}
%% 
\begin{alltt}
    peek\{\textnormal{\textit{frame-or-predicate-goal}}\}
\end{alltt}
%% 
The result is that the results of the query are returned without
re-executing the query.

At first glance, it may seem that this is no different from tabling, which
also returns the results without re-executing, but there is an important
and subtle difference: \texttt{peek\{Goal\}} returns the answers to all
\emph{previously executed} queries that unify with \emph{Goal}. There could
be several such previous queries or none.   
If none, then the above primitive returns no answers; if several, the
primitive returns the answers to all those queries. In contrast, if the
query were just \texttt{Goal} then, if this exact query was asked before then
its answers will be returned without re-evaluation. The answers to other
queries that unify with \texttt{Goal} will \emph{not} be returned. If
\texttt{Goal} was not asked before then it will be evaluated and the
results returned.

To illustrate, consider the following simple example:
%% 
\begin{verbatim}
    p(1,a), p(1,b).
    p(2,c), p(2,d).
    p(3,e). p(3,f).
    ?- p(1,?), p(2,?).
\end{verbatim}
%% 
If we now ask the query \texttt{peek\{p(?X,?Y)\}} then we get the following
four answers:
%% 
\begin{verbatim}
    ?X = 1
    ?Y = a

    ?X = 1
    ?Y = b

    ?X = 2
    ?Y = c

    ?X = 2
    ?Y = d
\end{verbatim}
%% 
Note that \texttt{?X=3} and \texttt{?Y=e}, \texttt{?Y=f} are not returned
because the query \texttt{p(3,?Y)} was never asked. In contrast, the query
\texttt{p(?X,?Y)} will return all six answers (and it will be evaluated,
as it is not a \texttt{peek}-query)  and after asking
\texttt{p(?X,?Y)} the subsequent query 
\texttt{peek\{p(?X,?Y)\}} will also return all six
answers (because it will pick up the answers produced by
\texttt{p(?X,?Y)}).


Another important use of the \texttt{peek} primitive is that it enables one
to look into the partially computed query results during pauses in the
computation. For instance, start \ERGO and run
\texttt{demo\{owl\_benchmark\}} at the prompt. After a few seconds, type
\texttt{Ctrl-C} to pause the execution and at the prompt ask the query
\texttt{?X::?Y}. Since non-transactional queries are currently
not allowed during these pauses (see Section~\ref{sec-runtime}),
one would see an error message like this:
%% 
\begin{verbatim}
   ++Abort[Ergo]> you are trying to query or modify the knowledge base while
	   a previous command is paused; only informational queries regarding
	   the system runtime state are permitted at this point
\end{verbatim}
%% 
However, \texttt{peek\{?X::?Y\}} \emph{is} allowed, and it will give
the answers to the query that have been computed so far.


%%% Local Variables: 
%%% mode: latex
%%% TeX-master: "../ergo-manual"
%%% End: 
