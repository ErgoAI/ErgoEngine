\section{Installation}\label{sec-install}

The \FLSYSTEM reasoner is part of the Coherent \ERGOAI
and comes bundled with \ERGOAI Studio and various \ERGO connectors (e.g., to
RDF, Java).
To install \ERGOAI on Windows, download the \ERGO installer,
\texttt{ergoAI.exe}, and
then click through the installation process.
Note that if you choose to install \ERGO in \texttt{Program Files} or some
other system directory, you must use an account with administrative
privileges not just during the installation but also when using \ERGO
afterwards. It is therefore generally recommended to install \ERGO in a
user-owned folder.

For Linux and Mac, the installer is \texttt{ergoAI.run}; it is a
self-extracting archive, which you can simply execute as
%% 
\begin{verbatim}
    sh ergoAI.run
\end{verbatim}
%% 
and \ERGOAI will be installed in a subdirectory called
\texttt{Coherent}. 

To start \FLSYSTEM via the UI, simply double-click on the \ERGOAI IDE icon
that should appear on your desktop after the installation.
If things go wrong on startup, the studio should
detect the problem and present an error dialog. If your email client is
configured properly, it will also
even draft an email bug report and offer you to send it.
If the startup was successful, but you encounter a problem later,
use the Studio's bug reporter. It is found in the listener window and is
accessible via the menu \texttt{Tools/Send Bug Report}. 

If for some reason you prefer to start the \FLSYSTEM reasoner as a
standalone engine on command line, there is and icon called \ERGOAI Terminal.
You can also locate the folder where the reasoner is
installed and type \texttt{runergo} there. See Section~\ref{sec-running}
for more details.

%%% Local Variables: 
%%% mode: latex
%%% TeX-master: "../flora2-manual"
%%% End: 

