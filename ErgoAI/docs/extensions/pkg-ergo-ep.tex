\chapter[Evidential Probabilistic Reasoning in  \FLSYSTEM]
{Evidential Probabilistic Reasoning in  \FLSYSTEM\\
  {\Large by Theresa Swift}}

\newcommand{\pct}{{\ensuremath{\tt \backslash{}pct}}}
\newcommand{\isa}{\,{\bf{:}}\,}
\newcommand{\subcl}{\,{\bf{::}}\,}

Evidential probability~\cite{KybTen:2001} is an approach to reasoning
about probabilistic information that may be approximate, incomplete,
or even contradictory. Rather than providing a full calculus for
probabilistic deduction, evidential probability addresses the question
of the probability of whether a given object is a member of a given
class.  To support this, evidential probability extends 
%the subclass declarations of 
\FLSYSTEM with {\em statistical statements} of
the form
\[
    \pct (targC,refC,Low,High)
\]
where $targC,~refC$ are \ERGO classes, while $Low$ and $High$ are
numbers between 0 and 1.  Such a statement indicates that any given
element of $refC$ is an element of $targC$ with probability between
$Lower$ and $Upper$.  For instance
\begin{center}
{\tt \pct(stolen,redRacing,0.0084,0.0476).}
\end{center}
could be used to indicate that the proportion of {\tt redRacing}
bicycles that are stolen in a given town is between $0.0084$ and
$0.476$.~\footnote{
  In \cite{KybTen:2001}, a more general model is
  presented, which addresses the question of whether a given $n$-tuple
  of domain elements is in the extension of a formula with $n$ free
  variables.}

In order to determine the probability of whether an individual $o$ is
in a class $C$ (when $o$ cannot be proved for certain to be in $C$)
statistical statements are used together with Ergo's class membership
(\isa/2) and subclass (\subcl/2) statements.  Information about the
classes to which $o$ certainly belongs is extended with statistical
information in the following manner.  A \emph{candidate} set $Cand$ is
collected by examining each statistical $\pct$-statement $S$ for which
$o$ is known to be an element of the reference class of $S$ and for
which $C$ is a subclass of the target class of $S$.  Namely,
\[
Cand = \{refC|\,\pct(targC,refC,Low,High),~C::targC,~o\in refC\}
\]
Using this candidate set, a series of rules is used to derive a single
interval representing the probability that $o \in targC$.  

As mentioned above, evidential probability is good for modelling
situations where probabilistic information may be missing or
inconsistent.  For instance, consider an individual {\em Mary} in a
given knowledge base.  {\em Mary} might belong to a number of
different classes: female, mother-of-2, American, resident-of-Virginia, over-40, college-educated, weekend-painter, and so on.  To
understand the likelihood that {\em Mary} would contract a given
well-studied disease, {\em d}, information for various epidemiological
studies could be consulted.  Some studies, such as those restricted to
male subjects, would not apply to Mary because she is not a member of the
reference class \emph{Man}.  On the other hand, some of
the classes to which {\em Mary} belongs, such as weekend painter, are
also irrelevant to whether she will contact $d$ --- this time because there
would be no $\pct$-facts with \emph{weekend-painter} as a reference class
(presumably because there would be no studies of the relationship between
painting on weekends to the disease in question).
Of the studies that do
pertain to {\em Mary}, some might be more relevant than others.  For
instance, a study of the incidence of $d$ for women over 35 would be
more relevant than a study of the general population because \emph{Mary}
belongs to the class \emph{over-40}, which is more specific than the class
of all persons.  At the same
time, various studies that pertain to {\em Mary} may conflict with one
another.  In general, we can't expect there to be a perfect study that
considers all potential risk factors for {\em Mary}.  Also, we can't
necessarily expect that information from the relevant studies is
entirely consistent, due to differences in experimental methods.
Thus, evidential probability combines the relevant information, weighs
some information more heavily than other information, and resolves
conflicts.

\paragraph{The Principles of Evidential Probability}
One means of weighing information is the principle of {\em
  specificity}: a statement $S_1$ may override statement $S_2$ if 1)
their associated intervals conflict (one interval is not contained in
the other); and 2) the reference class of $S_1$ is more specific to an
object $o_1$ than that of $S_2$.  A second principle is that of {\em
  precision}.  Given two intervals $(L_1,U_1)$ and $(L_2,U_2)$ where
one interval is retained in the other, only the more precise interval
is contained.  After repeatedly applying the principle of specificity,
then of precision, a final candidate set of intervals, $S_{fin}$ is
obtained.  The final probability is taken to be the smallest interval
containing all intervals in $S_{fin}$.

Evidential probability is thus not a full probabilistic logic, but a
meta-logic for defeasible reasoning about statistical statements once
non-probabilistic aspects of a model have been derived.  It is thus
more specialized and less powerful than other types of probabilistic
logics; but it is efficient to compute, and applicable to situations
where such logics don't apply, due to contradiction, incompleteness,
or other factors.~\footnote{Other prioritizations could also be
  considered, such as prioritizing more trusted information (say,
  information from better experiments or studies).  This type of
  priority is described in \cite{KybTen:2001} as {\em sharpening by
    richness}, but is not implemented here.}

\section*{Demonstration Example: Stolen Bikes}
The file {\tt
  .../Ergo/ergo\_demos/evidential\_probability/bikes.ergo}
provides an example of reasoning about evidential probability, and
contains a subclass hierarchy along with a set of statistical
statements.  To use evidential probability, first load the package
into the module {\tt ergo\_ep}:
%
\begin{verbatim}
ergo> [ evidential_probability >> ergo_ep].
\end{verbatim}
%
then load the example
%
\begin{alltt}
ergo> ['\textit{path-to}/ergo_demos/evidential_probability/bikes'].
\end{alltt}
%
On Windows, use double-backslashes instead of forward slashes:
%% 
\begin{alltt}
ergo> ['c:\textit{path-to}/ergo_demos/evidential_probability/bikes'].
\end{alltt}
%% 
At this stage, queries can be made about evidential probability.  The query:
%
\begin{verbatim}
ergo> \ep(stolen,redRacingImported,?L,?H)@ergo_ep.
\end{verbatim}
%
should return {\tt ?L = 0},{\tt ?H = 0.0454}.  We show in detail how these
bounds bounds were derived.  The first step is to {\em sharpen by
  specificity}, i.e., to collect all of the relevant statistical
statements that pertain to {\tt redRacingImported}, beginning with the
most specific.  There are no statistical statements about stolen
bicycles in the class {\tt redRacingImported}, but there are
statements for its immediate superclasses {\tt redRacing}, {\tt
  racingImported} and {\tt redImported}, all of which form the current
{\em candidate set}.  Next, we check statistical statements for the
immediate superclasses of the candidate set, namely {\tt red}, {\tt
  racing} and {\tt imported}.  Consider first the interval associated
with {\tt red}:{\tt [0.0084,0.0476]}.  This interval is considered to
conflict with that of e.g., {\tt redRacing}: {\tt [0,0.0454]} since
neither interval is contained in the other.  In this case, the
interval for {\tt red} is overridden and not considered further.
Similar considerations override intervals for {\tt imported} and
{\tt bike}.  Thus, at the end of sharpening by specificity, the
candidate classes and their intervals are:

{\tt redRacing:[0,0.0454], racing:[0,0.0467], redImported:[0,0.0467], racingImported:[0,0.0582]}.

The next step is to {\em sharpen by precision}, which throws out all
candidate intervals that are contained in other intervals.  This step
throws out all intervals except for that of {\tt redRacing}: [0,0.0454].

%%From this example, it can be seen that the statistical statements of
%%Evidential Probability essentially offer a way to probabilistically
%%extend the inheritance hierarchy of \ERGO.  The idea is to get
%%whatever information is relevant, but to prioritize information that
%%is more specific to an object more than general evidence, and in this
%%sense Evidential Probability has a defeasible flavor.  


%%% Local Variables: 
%%% mode: latex
%%% TeX-master: "../flora-packages"
%%% End: 
