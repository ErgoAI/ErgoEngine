
\chapter[Constraint Solving]
{Constraint Solving with MiniZinc\\
  \Large by Michael Kifer}



\section{Introduction}

MiniZinc is a uniform declarative constraint language that is understood by
most modern solvers for constraint and optimization problems. It comes bundled with a few such solvers:
\begin{itemize}
\item \texttt{gecode}: this is a top-notch solver---very powerful and fast.
  It is an excellent choice for finite domain constraints.
  A disadvantage is insufficient support for floating point numbers.
\item  \texttt{osicbc}: this solver has a much better support for the
  floats than \texttt{gecode}. In particular, it is a much better choice
  for linear programming type of problems. On the other hand, it may be
  slower than \texttt{gecode} for finite domains.  
\item \texttt{chuffed}: yet another solver for finite domains; it has no
  support for floats at all. This solver is more limited than the other
  two, but might be faster on some particular problems. Since generally no
  solver is uniformly better than all others, it is a good idea to try
  different solvers and see which one works best for a problem at hand.
\end{itemize}
%% 
Other solvers,
including most of the newest ones, such as Google OR-Tools,\footnote{
  \url{https://developers.google.com/optimization/}
  }
can be downloaded separately and installed as plugins.

This chapter assumes that the reader is familiar with the basics of MiniZinc.
The MiniZinc language is described in
\url{https://www.minizinc.org/doc-2.2.3/en/index.html}; see, especially, the
tutorial. 
The \FLSYSTEM interface to MiniZinc comes with several sample problems from that
tutorial, which are found in \url{.../ErgoAI/ergo_demos/minizinc/}.
The file \url{.../ErgoAI/ergo_demos/minizinc_examples.P} contains examples of
invocations of those problems from within \FLSYSTEM. After the
installation, all examples can be run together
by simply starting \FLSYSTEM and then loading the aforesaid file:
%% 
\begin{verbatim}
   ergo> ['ergo_demos/minizinc_examples'].
\end{verbatim}
%% 
Annotated results will be printed on the console.

\section{Installation}

Some Linux distributions (e.g., Ubuntu) come with ready-made MiniZinc .deb
or .rpm packages. However, one must make sure that the command
\url{minizinc} is provided by those packages, as some provide only the IDE.
Mac packages are also available.

In case a Linux or a Mac package is incomplete (or if one uses Windows),
MiniZinc can be downloaded from
\url{https://www.minizinc.org/software.html} --- it does not come with
\FLSYSTEM.

Once installed, make sure that the command \texttt{minizinc} is understood when typed in a
command window. If not, add \emph{folder-to-where-}\textbf{/bin/}minizinc is sitting to the
environment variable \texttt{PATH}.  In Windows, this is best done in
Control Panel; in Linux and Mac, add the command
%% 
\begin{verbatim}
    export PATH=$PATH:path-to-minizinc
\end{verbatim}
%% 
to \texttt{.bashrc} or an equivalent place. For instance,
%% 
\begin{verbatim}
    PATH=$PATH:$HOME/minizinc/MiniZincIDE-2.2.3-bundle-linux/bin
\end{verbatim}
%% 

\section{Loading the API}\label{sec-api-mod}

To load \FLSYSTEM's API for MiniZinc, just load the corresponding package
into some \FLSYSTEM's module. For instance,
%% 
\begin{verbatim}
    ?- [minizinc>>mzn].
\end{verbatim}
%% 
loads the API into the \FLSYSTEM module \texttt{mzn} and the calls
described below become available in this module.

\section{The API}

Constraint and optimization problems are specified using the MiniZinc
language in \emph{model files}, which have the suffix \texttt{.mzn}.
Such problems usually have a number of \emph{input variables}   and several
\emph{output} (or \emph{decision}) \emph{variables}.   
In principle, input values can be specified in the model file itself, but
this is generally not a good idea because typically one wants to solve
the same problem with different inputs.
For this reason, MiniZinc allows one to use models (that have no input)
with one or more \emph{input files}, which have the extension \texttt{.dzn}. 
In addition, the API to MiniZinc lets one pass parameters in-line, as
part of an \FLSYSTEM call.

\noindent
\emph{Important:} the MiniZinc model files (.mzn) must \textbf{not} have
\texttt{output} statements in them. Otherwise, errors and wrong answers
may result. (These output statements will be added automatically based on
output templates described below.)  


The API itself mainly consists of the following calls, which live in the
\FLSYSTEM module into which the API is loaded, as explained in
Section~\ref{sec-api-mod}. 
In the examples below we assume that this module is \texttt{mzn}, so, for
example, \texttt{solve/8} below should be invoked as a query as
\texttt{solve(...,...,...,...,...,...,...,...)@mzn}.   

%% 
\begin{itemize}
\item
  \texttt{solve(+MznF,+DatFs,+InPars,+Solver,+Solns,+OutTempl,-Rslt,-Xceptns)}: 
  Pluses here denote input parameters and minuses denote output parameters.
  The meaning of the arguments is as follows:
  %% 
  \begin{itemize}
  \item    \texttt{MznF}: should be bound to a path to the desired 
    model file (.mzn file) that contains a specification of a constraint or
    optimization problem. The path must be represented as a Prolog atom and
    can be absolute or relative to the current directory.
  \item \texttt{DatFs}: a list of paths (relative or absolute)  
    to the data files that describe all or some of the input parameters for
    the model in \texttt{MznF}. All paths must be atoms.
    If no data files are needed, just use the empty list \texttt{[]}. 
  \item \texttt{InPars}: a list of the \emph{in-line} input parameters to
    the model.
    These are supported for flexibility, to allow initialization of some or
    all input parameters directly in Prolog.
    Each parameter in \texttt{InPars} must have the form \emph{id} =
    \emph{value}, where \emph{id} must be the Id of an input variable used in
    the model file \texttt{MznF} and value must be a term understood by MiniZinc.
    Typically, such a term would be a number, an atom, or a function term.
    For instance,  \texttt{foo='[|1, 0|3, 4|8, 9|]'} would initialize the input
    variable \texttt{foo} with a 2-dimensional array of numbers;
    \texttt{Item = anon\_enum(8)}  would initialize the MiniZinc variable
    \texttt{Item}  of an enumerated type to a set
    \texttt{\{Item\_1,Item\_2,...,Item\_8\}}.
  \item \texttt{Solver}: the name of the solver to use. If this is a variable,
    the default solver \texttt{gecode} is used. 
  \item \texttt{Solns}: the number of solutions to show. 
    Could be \texttt{all} or a positive integer.
    Note: for optimization problems where a function is maximized or minimized,
    \texttt{Solns} \underline{must} be bound to 1. Otherwise, solvers would
    return also non-optimal solutions, and the desired optimal one may not
    be the first.
  \item \texttt{OutTempl}: the template showing how the output should look
    like. It has the form \emph{predname}($OutSpec_1$, ...,$OutSpec_n$)
    where \emph{predname} is the name of a predicate
    where the results will be stored (see below)
    and each $OutSpec$ can have one of these forms:
    %% 
    \begin{itemize}
    \item  An \emph{atom} representing an output (``decision'') variable
      defined in the model file \texttt{MznF} or an atom representing an
      arithmetic expression (in the syntax of MiniZinc, which is close to
      \FLSYSTEM) involving one or more decision variables.
      In the result, this atom will be replaced with the value of that
      variable or expression. Example: 'P*100.0'.
    \item A simple arithmetic expression involving decision variables,
      numbers, and +, $-$, *, /.  Example: P*100+9. The difference with the
      above is that the single quotes are omitted, for convenience.
    \item  A term of the form \texttt{+}(\emph{atom}) or
      \texttt{str}(\emph{atom}).  In the result, this will be replaced with
      \emph{atom} verbatim. Note: if \texttt{atom} is not alphanumerical,
      it must be quoted.

    \item A list of the form [$OutSpec_1$, ...,$OutSpec_n$]. Each
      \emph{OutSpec} in the list must have one of the forms
      described here.
    \item A term of the form $OutSpec_1$ \texttt{=} $OutSpec_2$. Each
      \emph{OutSpec} must have one of the forms
      described here.
    \end{itemize}
    %% 
    The template predicate cannot be \texttt{solve/8}, \texttt{solve\_flex/8},
    \texttt{show/1}, \texttt{delete/1},  
    and the arity must be greater than 0.  
  \item \texttt{Rslt}: must be a term that matches the output
    template---typically just a variable.  Solutions to the constraint and
    optimization problems will be returned as bindings to variables in this
    term.  The term cannot match the predicates \texttt{solve/8},
    \texttt{solve\_flex/8}, \texttt{show/1}, or \texttt{delete/1} (e.g.,
    cannot be \texttt{solve(?,?,?,?,?,?,?,?)}), and the arity must
    be greater than 0 (i.e., no predicates like \texttt{foo/0}).

    \emph{Note}: solutions are also saved internally and
    can be used
    at a later analysis stage. If any of these solutions are no longer
    needed, they can be emptied out, as described below.

  \item \texttt{Xceptns}: a list of exceptions returned by the solver.
    If the constraint/optimization problem was solved successfully, this
    variable will be bound to an empty list \texttt{[]}. Otherwise,
    each exception has the form
    \texttt{(reason=...,model\_file=...)}  and the
    following reasons might be returned:
    %% 
    \begin{itemize}
    \item  \texttt{unsatisfiable} --- the optimization problem is
      unsatisfiable.
    \item \texttt{unbounded}  --- the optimization problem has an unbounded
      objective function. For example, if the problem is to maximize the
      function and the function has no maximum (under the given
      constraints); or the problem is to minimize the function, and there
      is no minimum.
    \item \texttt{unsatisfiable\_or\_unbounded} --- one of the above.
    \item \texttt{unknown} --- could not find a solution within the limits
      (e.g., timeout).
    \item \texttt{error} --- search resulted in an error. 
    \end{itemize}
    %% 
  \end{itemize}
  %% 
  Note: exceptions are separate from other kinds of errors, such as a
  syntax errors in a model or data file or using a solver with a feature or
  option it does not support. Errors are explained later.
\item
  \texttt{solve\_flex(+MznF,+DatFs,+InPars,+Solver,+Solns,+OutTempl,-Rslt,-Xceptns)}:
  The meaning of the parameters is the same as for \texttt{solve/8}.  The
  difference is that if \texttt{InPars} or \texttt{OutTempl} is non-ground,
  the call to MiniZinc is delayed until they both become ground. If the top
  query finishes and one of them is still not ground, an error will be
  issued.  This version is used in cases when it is hard to estimate when
  \texttt{InPars} or \texttt{OutTempl} may become ground, so calls to
  \texttt{solve\_flex} can be placed early without worrying that
  \texttt{InPars} and \texttt{OutTempl} may not have been computed
  yet. But, of course, one must ensure that \texttt{InPars}  and
  \texttt{OutTempl}  will eventually
  get bound---or deal with the error.
\end{itemize}
%% 

\paragraph{How does MiniZinc return complex structures back to \FLSYSTEM?}
MiniZinc has a number of data structures that do not have direct
equivalence in \FLSYSTEM, so they are mapped to HiLog terms when results are
returned as bindings for the \texttt{Rslt} variable. Here is the correspondence:
%% 
\begin{itemize}
\item  Numbers are passed back as integers and floats.
\item  MiniZinc strings and identifiers are passed as atoms (\FLSYSTEM
  symbols).
\item  Arrays are returned as lists. Multi-dimensional arrays are flattened
  and passed as lists as well. For instance, a 2-dimensional array
  \texttt{[|1, 2|3, 4|5, 6|]} will be returned as \texttt{[1,2,3,4,5,6]}.
\item Sets are returned as terms of the form
  \texttt{\{\}((elt1,elt2,elt3,...))}. 
  For example, the set \texttt{\{a,b,c\}} comes back as the term
  \texttt{\{\}((a,b,c))}. Note that in \FLSYSTEM this term is really
  \texttt{'\{\}'((a,(b,c)))}. Observe the double parentheses, which indicate
  that the functor symbol here is \texttt{\{\}/1}, not \texttt{\{\}/3}.
\item A MiniZinc range expressions of the form \texttt{N1..N2} is returned as
  the term \texttt{N1..N2}.
\end{itemize}
%% 

\paragraph{Accessing solutions at a later time.}

Solutions obtained via \texttt{slove/8} or \texttt{solve\_flex/8} 
can be retrieved later via a call to \texttt{show/1}.  For instance, if
\texttt{solve/8} returns the solutions as a term \texttt{foo/2} then
they cal later also be obtained by querying
%% 
\begin{verbatim}
   ?- show(foo(?X,?Y))@mzn.
\end{verbatim}
%% 
If \texttt{solve/8} was also called to solve another constraint problem,
returning the results in a predicate \texttt{bar/3} then they can also be
retrieved and, in fact, the two sets of solutions can be combined. For
instance, 
%% 
\begin{verbatim}
   ?- show(foo(?X,?Y))@mzn, show(bar(?X,?V,?W))@mzn, ?Y < ?W.
\end{verbatim}
%% 
If any set of solutions is no longer needed, it can be deleted.
For instance,
%% 
\begin{verbatim}
   ?- delete(foo(?,?))@mzn.
\end{verbatim}
%% 

\paragraph{Errors vs. exceptions.}
If a model or data file contains a syntax error or a feature that the
chosen solver does not support, the \texttt{solve/8} and
\texttt{solve\_flex/8} calls will fail and a message will be printed to
standard output:
%% 
\begin{verbatim}
    +++ MiniZinc: syntax or type errors found; details in ....some file...
\end{verbatim}
%% 
The user will then be able to find the details about the problem.

Note: errors are different from exceptions. If only exceptions are returned
(and no errors), the calls \texttt{solve/8} and \texttt{solve\_flex/8} will
succeed. In contrast, they will fail in case of errors.

\paragraph{Debugging API.}
For further development and bug reporting, the following calls are useful.
They are all 0-ary predicates that take no arguments.
%% 
\begin{itemize}
\item  \texttt{keep\_tmpfiles}: The MiniZinc interface creates a number of
  temporary files, which are deleted, if MiniZinc finished normally and
  without an error. However, if a bug is suspected, it is desirable to
  preserve these files and send them to the developers. 
  This can be achieved by executing the query
  \texttt{keep\_tmpfiles@mzn}
  before the call to \texttt{solve/8}.
\item \texttt{show\_mzn\_cmd}: Executing this as a query will cause the
  \texttt{solve/8} predicate, described earlier, to print the shell command
  that was used to invoke MiniZinc in each call.
  This is useful if one suspects a bug in the API.
\item \texttt{dbg\_clear}: Executing this clears out the flags set by the
  above debugging calls. As a result, the temporary files will again be
  deleted after each invocation of MiniZinc and shell commands will not be
  shown.
\end{itemize}
%% 



%%% Local Variables: 
%%% mode: latex
%%% TeX-master: "../ergo-packages"
%%% End: 
