  \index{$\mid\mid$ operator}
  \index{operator!$\mid\mid$}
  \index{$++$ operator}
  \index{operator!$++$}
  \index{$-\,-$ operator}
  \index{operator!$-\,-$}
  \index{$\&\&$ operator}
  \index{operator!$\&\&$}
  In \ERGO, \texttt{\bs{}is/2} also understands
  the string concatenation operator, $\mid\mid$,
  and the list append, difference, and intersect operators \texttt{++},
  \texttt{-\,-}, and \texttt{\&\&}. 
  For instance,
  %% 
  \begin{quote}
    {\tt
   ?- ?X \bs{}is abc $\mid\mid$ 'cde: ' $\mid\mid$ f(?P) $\mid\mid$  ' ;',
   writeln(?X)@\bs{}io.
   }
  \end{quote}
   writes out this:
  \begin{quote}
   \texttt{abccde: f(?A) ;}
  \end{quote}
  %% 
  Note that this operator converts terms into their printable
  representation and tries to give meaningful names to variables.

  Examples of list append/difference/intersection include:
  %% 
\begin{alltt}
   ?- ?X \bs{}is (([a,c,?L,b]-\,-[c])++[?L,1]) -\,- [?L,a].\\
   {\it Result\/}: ?X = [b, 1]\\

   ?- ?X \bs{}is [a,c,?L,b]-\,-[c]++[?L,1].\\
   {\it Result\/}: ?X = [a, b]\\

   ?- ?X \bs{}is [a,b]++[c]++[?L,1].\\
   {\it Result\/}: ?X = [a, b, c, ?\_h9660, 1]\\

   ?- ?X \bs{}is [a,b]++[c]++[p,1]&&[b,c,p].  \\
   {\it Result\/} :?X = [a, b, c, p]
\end{alltt}
  %% 
  To understand the results of these queries, the reader needs to keep in
  mind that
  \texttt{++}, \texttt{-\,-}, and \texttt{\&\&} 
  are \emph{right-associative} operators. In the first
  example, we explicitly placed the parentheses to change the order of the
  operations, but in the last three examples parenthesizing happens
  implicitly, through right-association. For example, in the second case we
  really have \texttt{[a,c,?L,b]-\,-([c]++[?L,1])}. The other important thing
  to note is that variables are not unified, but are compared via lexical
  identity. This is why, in the second query, \texttt{?L} disappears from
  the result, but if the list expression were
  \texttt{[a,c,?L,b]-\,-([c]++[?K,1])} then the result of the evaluation
  would be \texttt{[a,?L,b]}. 

%%% Local Variables: 
%%% mode: latex
%%% TeX-master: "../flora2-manual"
%%% End: 
