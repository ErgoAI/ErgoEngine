\chapter[Importing JSON Structures]
{Importing and Exporting JSON Structures\\
  {\Large by Michael Kifer}}

JSON is a popular notation for representing data. It is defined
by the ECMA-404 standard, which can be found at \url{http://www.json.org/}.
This chapter describes the \ERGO facility for importing JSON structures
called \emph{values}; it is based on an open source parser called Parson
\url{https://github.com/kgabis/parson}.

\section{Introduction}

In brief, a JSON structure is a \emph{value}, which can be
  an \emph{object}, an
\emph{array}, a \emph{string}, a \emph{number},  \texttt{true}, \texttt{false},
or \texttt{null}. An array is an expression of the form
\texttt{[$value_1$, ..., $value_n$]}; an object has the form
\texttt{\{ $string_1$ : $value_1$,  ..., $string_n$ : $value_n$
  \}}; strings are enclosed in double quotes and are called the \emph{keys}
of the object; numbers have the usual
syntax, and \texttt{true}, \texttt{false}, and \texttt{null} are constants
as written. Here are examples of relatively simple JSON values:
%% 
\begin{verbatim}
{
  "first": "John",
  "last": "Doe",
  "age": 25
}

[1, 2, {"one" : 1.1, "two": 2.22}, null]

123
\end{verbatim}
%% 
and here is a more complex example where values are nested to the depth of
five:
%% 
\begin{verbatim}
{
  "status": "ok",
  "results": [{"recordings": [{"id": "12345"}],
               "score": 0.789,
               "id": "9876"
              }]
}
\end{verbatim}
%% 

Although not part of the standard, it is quite common to see JSON
structures that contains comments like in C, Java, etc. The multiline
comments have the form \texttt{/* ... */} and the here-to-end-of-line
comments start with the \texttt{//}. The JSON parser ignores such comments.

The standard recommends, but does not require, that the keys in an object
do not have duplicates (at the same level of nesting). Thus, for instance,
%% 
\begin{verbatim}
    {"a":1, "b":2, "b":3}
\end{verbatim}
%% 
is allowed, but discouraged. By default, the JSON parser does not allow
duplicate keys and considers such objects as ill-formed. However, it also
provides an option to allow duplicate keys.


\section{API for Importing JSON  as Terms}

When \ERGO ingests a JSON structure, it represents it as a term as follows:
%% 
\begin{itemize}
\item  Arrays are represented as lists.
\item  Strings are represented as \ERGO symbols (Prolog atoms).
\item  Numbers are represented as such.
\item \texttt{true}, \texttt{false}, \texttt{null} are represented as the
  \texttt{\bs{}true}, \texttt{\bs{}false}, and \texttt{\bs{}@?}  (to
  remind: \texttt{\bs{}@?} is the \FLSYSTEM quasi-constant used to
  represent null values).
\item Finally, an object of the form \texttt{\{
    $str_1$:$val_1$,...,$str_n$:$val_n$\}} is represented as
  \texttt{json([$str_1'$=$val_1'$,...,$str_n'$=$val_n'$])}, where
  $str_i'$ is the atom corresponding to the string $str_i$ and $val_i'$ is
  the \ERGO representation of the JSON value $val_i$.
  Here \texttt{json} is a unary Prolog, not HiLog, function
  symbol.
\end{itemize}
%% 
For instance, the above examples would be represented as HiLog 
\ERGO terms as follows:
%% 
\begin{verbatim}
json([first = John, last = Doe, age = 25])
[1, 2, json([one = 1.1000, two = 2.2200]), \@?]
123
json([status = ok,
      results = [json([recordings = [json([id = '12345'])],
                       score = 0.7890,
                       id = '9876']
                     )]
     ])
\end{verbatim}
%% 
where we tried to pretty-print the last result so it would be easier to
relate to the original (which was also pretty-printed).

\ERGO provides the following methods for importing JSON:
%% 
\begin{itemize}
\item
  \texttt{\emph{Source}[parse -> ?\emph{Result}]@\bs{}json}\\
  Here \emph{Source} can have one of these forms
  %% 
  \begin{itemize}
  \item   \texttt{string(\emph{Atom})}  
  \item   \texttt{str(\emph{Atom})}
  \item   \texttt{url(\emph{Atom})}
  \item   \texttt{file(\emph{Atom})}
  \item   \emph{Atom}
  \item   a variable
  \end{itemize}
  %% 
  The forms \texttt{string(\emph{Atom})} and \texttt{str(\emph{Atom})}
  must supply an atom whose content is a JSON structure and
  \emph{Result} will then be bound to the \ERGO representation of that
  structure.
  The form \texttt{url(\emph{Atom})} can be used to ask \ERGO to get a JSON
  document from the Web. In that case, \emph{Atom} must be a URL. 
  The forms \texttt{file(\emph{Atom})} and \emph{Atom}
  interpret \emph{Atom} as a file name and will read the JSON structure
  from there. The last form, when the source is a variable, assumes
  that the JSON structure will come from the standard input. The user will
  have to send the end-of-file signal (Ctrl-D in Linux or Mac; Ctrl-Z in
  Windows) in order to tell when the entire term has been entered.\footnote{
    Sending the end-of-file signal is not possible in the \ERGOAI Studio Listener,
    so this last option is not available through the studio.
  }
  If the input JSON structure contains a syntax error or some other problem
  is encountered (e.g., not enough memory) then the above predicate will
  fail and a warning indicating the reason will be printed to the standard
  output.

  \emph{?Result} can be a variable or any other term. If  \emph{?Result}
  has the form \texttt{pretty}(\emph{?Var})  then \emph{?Var}
  will get bound to a pretty-printed string representation of the input
  JSON structure. If \emph{?Result} has any other form (typically a
  variable) then the input is converted into an \ERGO term as explained
  above.
  For instance, the query 
  \texttt{string('\{"abc":1, "cde":2\}')[parse->?X]@\bs{}json}   
  will bind \texttt{?X} to the \ERGO HiLog term \texttt{json([abc=1,cde=2])}
  while the query  \texttt{string('\{"abc":1,
    "cde":2\}')[parse->pretty(?X)]@\bs{}json}  will bind \texttt{?X} to the atom
  %% 
\begin{verbatim}
'{
    "abc": 1,
    "cde": 2
}'
\end{verbatim}
  %% 
  which is a pretty-printed copy of the input JSON string.
\item
  \texttt{\emph{Source}[parse(\emph{Selector}) -> ?\emph{Result}]@\bs{}json}\\
  The meaning of \emph{Source}  and \emph{Result}  parameters here are the
  same as before.
  The \emph{Selector} parameter must be a path expression of the form
  ``string1.string2.string3'' (with one or more components) that allows
  one to select the \emph{first} sub-object of a bigger JSON object and
  return its
  representation. Note, the first argument \emph{must} supply an object, not an
  array or some other type of value. For instance, if the input is
  %% 
\begin{verbatim}
{ "first":1, "second":{"third":[1,2], "fourth":{"fifth":3}} }  
\end{verbatim}
  %% 
  then the query \texttt{?[parse(first) -> ?X]@\bs{}json} will bind
  \texttt{?X} to 1 while
  \\
  \texttt{?[parse('second.fourth') -> ?X]@\bs{}json} will bind it to
  \texttt{json([fifth = 3])}.

  Note that the selector lets one navigate through subobjects but not
  through arrays. If an array is encountered in the middle, the query will
  fail. For instance, if the input is
\begin{verbatim}
{ "first":1, "second":[{"third":[1,2], "fourth":{"fifth":3}}] }  
\end{verbatim}
  then the query \texttt{?[parse('second.fourth') -> ?X]@\bs{}json}
  will fail and \texttt{?X} will not be bound to anything because the
  selector \texttt{"second"} points to an array and the selector
  \texttt{"fourth"} cannot penetrate it. 

  Also note that if the JSON structure has more than one sub-object that
  satisfies the selection and duplicate keys are allowed
  (e.g., in \texttt{\{"a":1, "a":2\}} both 1 and 2 satisfy the selection)
  then only the first sub-object will be returned. (See below to learn about
  duplicate keys in JSON.)

\item \texttt{set\_option(\emph{option}=\emph{value})@\bs{}json}\\
  This sets options for parsing JSON for all the subsequent calls to the
  \texttt{\bs{}json} module. Currently, only the following is supported:
  %% 
\begin{verbatim}
    duplicate_keys=true
    duplicate_keys=false
\end{verbatim}
  %% 
  As explained earlier, the default is that duplicate keys in JSON objects
  are treated as syntax errors. The first of the above options tells the
  parser to allow the duplicates. The second option restores the default.
\end{itemize}
%% 

Here is a more complex example, which uses the JSON parser to
process the result of a search of
Google's Knowledge Graph to see what it knows about Benjamin Grosof.
To make the output a bit more manageable, we are only asking to get the
JSON subobject rooted at the property \texttt{itemListElement}.
The Knowledge Graph itself is queried using XSB's \texttt{curl}
library. 
(The Google KG's
session key in the example is invalid: one must supply one's own key.)
%% 
\begin{alltt}
?- ?U = '\url{https://kgsearch.googleapis.com/v1/entities:search?query=benjamin_grosof&key=XYZ&limit=1}',
   load_page(url(?U),
             [secure(false)], ?, ?_SearchResult, ?)@\bs{}plgall(curl),
   str(?_SearchResult)[parse(itemListElement) -> ?Answer]@\bs{}json.
\end{alltt}
%% 
The answer to this query is
%% 
\begin{alltt}
?Answer = [json(['@type' = EntitySearchResult,
                 result = json(['@id' = 'kg:/m/09pb9y8',
                                name = 'Benjamin Nathan Grosof',
                                '@type' = [Person, Thing],
                                description = Mathematician]),
                 resultScore = 19.3944])]
\end{alltt}
%% 

The same can actually be obtained in a much simpler way using the
\texttt{url} feature for the JSON source, as described above:
%% 
\begin{alltt}
?- ?U = '\url{https://kgsearch.googleapis.com/v1/entities:search?query=benjamin_grosof&key=XYZ&limit=1}',
   url(?U)[ parse(itemListElement) -> ?Answer @\bs{}json.
\end{alltt}
%% 
At present, the \texttt{url(...)} feature works only for documents
that are not protected by passwords or SSL. 

\section{API for Importing JSON as Facts}

The API for importing JSON as terms is useful if one
needs to traverse the imported JSON tree structure and process it in some
complex way. However, in knowledge
interchange, JSON is often used to exchange facts about enterprises being
modeled by the different knowledge base. For instance, the native
representation in Wikidata and MongoDB is JSON and to get the Wikidata or the
MongoDB facts into \ERGO we would want to represent the information as
queriable facts.
Fortunately, converting JSON into \ERGO facts is easy because the former is
mappable 1-1 to \ERGO frames. For instance, the following JSON
%% 
\begin{verbatim}
   {"kind": "person", "fullName": "John Doe", "age": 22, "gender": "Male",
    "child": {{"fullName":"Bob Doe", "age":1},    // embedded JSON objects
              {"fullName":"Alice Doe", "age":3}},
    "citiesLived": [{ "place":"Boston", "numberOfYears":5}, // JSON objects
                    {"place":"Rome", "numberOfYears":6}]}   // embedded in list
\end{verbatim}
%% 
translates into this:
%% 
\begin{verbatim}
   \#[kind->person, fullName->'John Doe', age->22, gender->Male,
      child->{\#[fullName->'Bob Doe', age->1],
              \#[fullName->'Alice Doe', age->3]},
      citiesLived->[\#[place->Boston, numberOfYears->5],
                    \#[place->Rome, numberOfYears->6]]
   ].
\end{verbatim}
%% 
The principle of this translation
should be obvious from the above example except that frames
are not allowed inside lists, and so
%% 
\begin{verbatim}
   [\#[place->Boston,numberOfYears->5], \#[place->Rome,numberOfYears->6]]
\end{verbatim}
%% 
is not a valid \ERGO 
syntax. However, this problem is side-stepped by converting 
lists with embedded frames, such as above,
into plain lists augmented with additional
frame-facts. For instance, the above offending list is represented as
%% 
\begin{verbatim}
   [newObjId1, newObjId2]                  // complex list became plain list
\end{verbatim}
%% 
where \texttt{newObjId1}  and \texttt{newObjId2} 
are newly invented constants
that do not appear anywhere else. In addition, the following facts are
added:
%% 
\begin{verbatim}
   newObjId1[place->Boston,numberOfYears->5]. // these are the facts that were
   newObjId2[place->Rome,numberOfYears->6].   // embedded in the above list
\end{verbatim}
%% 
Thus, the actual translation of the JSON structure in question is
%% 
\begin{verbatim}
   \#[kind->person, fullName->'John Doe', age->22, gender->Male,
      child->{\#[fullName->'Bob Doe', age->1],
              \#[fullName->'Alice Doe', age->3]},
      citiesLived->[newObjId1, newObjId2]  // list no longer has embedded frames
   ].
   newObjId1[place->Boston, numberOfYears->5].  // frames formerly
   newObjId2[place->Rome, numberOfYears->6].    // embedded in a list
\end{verbatim}
%% 

Conversion of JSON structures into facts is done by the following API calls:
%% 
\begin{itemize}
\item  \texttt{?Src[parse2memory(?Mod)]@\bs{}json}:
  The meaning of \texttt{?Src} is as before. This API call takes the input
  JSON structure, which must be a JSON object (and not a list, number,
  etc.) and inserts facts, as explained above, into the \ERGO module
  \texttt{?Mod}.
\item \texttt{?Src[parse2memory(?Mod,?Selector)]@\bs{}json}:
  Like the previous call but also takes the selector argument whose meaning
  is as in the case of the term-based JSON import.
\item \texttt{?Src[parse2file(?File)@\bs{}json]}:
  This is similar to parsing to memory, but the facts are instead written
  to the specified file. If the file already exists, it is erased first.
  The file can then be loaded or added into some \ERGO module (adding is
  recommended).
\item \texttt{?Src[parse2file(?File,?Selector)@\bs{}json]}:
  Like the previous case, but also takes the selector argument.
\end{itemize}
%% 
Only JSON objects (i.e., \{...\} - structures, not standalone constants or
lists) can be converted to facts.

Conversion to facts involves creation of a root \ERGO object that represents
the entire JSON structure. This object then points to
other objects that represent the various components of that structure, etc.
The Id of the root object can be obtained via the query
%% 
\begin{verbatim}
   ?- ?Module[json_root->?Oid]@\json.
\end{verbatim}
%% 
Here \texttt{?Module}   is the module into which the JSON object is
dumped by the\texttt{parse2memory} method or into which the file of facts is loaded 
when \texttt{parse2file} is used.
If more than one JSON object is dumped into the same module, the above query
will return multiple answers---one for each JSON structure dumped into the module.
One can ``forget'' the root-level oids using this API call:
%% 
\begin{verbatim}
   ?- ?Module[forget_roots]@\json.
\end{verbatim}
%% 
This is useful in situations when one is done processing a previous
JSON structure and needs to traverse a newly-dumped structure into the same
module.
However, the most common way of working with JSON is when modules hold just
one JSON structure at a time,
and \texttt{erasemodule\{...\}} is used before another JSON
structure is dumped into the same module. 


\section{Exporting to JSON}

\ERGO provides API for exporting HiLog terms as well as objects to JSON.

\subsection{Exporting  HiLog Terms to JSON}

The case of terms is simple: a term is represented simply as a JSON object
with two features: \emph{functor} and \emph{arguments}. The functor is also
a term so it is futher converted according to the same rules.  The
\emph{arguments} part is a list of terms and the latter are converted
recursively by the same rule. For instance,
%% 
\begin{verbatim}
   ergo> p(o)(${a(9)},b,?L,[pp(ii),2,3,?L])[term2json -> ?J]@\json.
   ?J = '{"functor":{"functor":"p","arguments":["o"]}
          "arguments":[{"predicate":"a","module":"main","arguments":[9]},
                       "b",
                       {"variable":"h0"},
                       [{"functor":"pp","arguments":["ii"]},
                       2,
                       3,
                       {"variable":"h0"}]]}'

   ergo> foo(a,b,bar(c,d))[term2json->?J]@\json.
   ?J = '{"functor":"foo",
          "arguments":["a","b",{"functor":"bar","arguments":["c","d"]}]}'

   ergo> \foo("2010-12-12"^^\date, "kkk"^^\mytype, 123)[term2json -> ?J]@\json.
   ?J = '{"functor":"\\\\foo",
          "arguments":[
              {"datatype":"\\\\date","literal":"2010-12-12"},
              {"datatype":"\\\\mytype","literal":"kkk"},
              123
           ]}'

   ergo> [a,b,bar(c,d)][term2json->?J]@\json.
   ?J = '["a","b",{"functor":"bar","arguments":["c","d"]}]'

   ergo> (a,b,bar(c,d))[term2json->?J]@\json.
   ?J = '{"commalist":["a","b",{"functor":"bar","arguments":["c","d"]}]}'
\end{verbatim}
%% 
Observe that \verb|\\\\foo| and the like above represents \texttt{\bs{}foo}
with the backslash escaped with another backslash, as required by
JSON. \ERGO shows four backslashes, as its output convention is to double
every backslash so there would be no confusion between, say,
\texttt{\bs{}n} representing the newline character and \texttt{\bs{}\bs{}n}
representing a string consisting of a backslash and a letter \texttt{n}.

Note that a term can be a reified predicate in which case the
\texttt{"predicate"} feature name is used instead of \texttt{"functor"}.
Also, a variable is translated into a JSON object of the form
\texttt{\{"variable": "\textnormal{\emph{varname}}"\}}. Since variable names in a logic formula
are immaterial and all that matters is whether two variables are the same
or not, only internal names are shown. In the above example, the two
occurrences of \texttt{?X} are shown as \texttt{"h0"}.
Frame and subclass/isa formulas are also supported, but not aggregate functions.
To see whether a particular form of a reified formula is supported
and how it is represented in JSON, use the JSON API method \texttt{term2json},
as shown above.
The general form of that method is given below:
%% 
\begin{itemize}
\item  \texttt{?Term[term2json -> ?Json]@\bs{}json} --- convert HiLog term
  \texttt{?Term} into a JSON expression.   The result is
  an atom (an \ERGO symbol) that contains the JSON expression. Such an atom
  can be sent to a JSON-aware external application.
\end{itemize}
%% 

\subsection{Exporting \ERGO Objects to JSON}

This API takes a HiLog term that is interpreted as an object Id and returns
the JSON encoding of all the immediate superclasses of that object and all the
properties of that object.
%% and for every descended object it encodes the same information.
The input object can be in the current module or in some
other module. Furthermore, the API can take conditions that would filter
out the properties of the object that we are looking for as well as
eliminate the descendant object that we don't want to see in the JSON encoding.
The idea of the encoding can best be understood via examples.

The first example gives a JSON encoding for the object \texttt{kati} from
the \texttt{family\_obj.flr} demo located in the \texttt{demos/} folder in
the \ERGO distribution.   First, we need to load this demo via the command
\texttt{demo\{family\_obj\}}. To get the JSON encoding, we use the
\texttt{object2json} method and then pretty-print the result  as explained
previously. That is,
%% 
\begin{verbatim}
    ?- demo{family_obj},
       set_option(duplicate_keys=true)@\json,
       kati[object2json -> ?Json]@\json,
       string(?Json)[parse->pretty(?Res)]@\json,
       writeln(?Res)@\io.

    {
            "\\self": "kati",
            "\\isa": [
                "female"
            ],
            "ancestor": "hermann",
            "ancestor": "johanna",
            "ancestor": "rita",
            "ancestor": "wilhelm",
            "brother": "bernhard",
            "brother": "karl",
            "daughter": "eva",
            "father": "hermann",
            "mother": "johanna",
            "parent": "hermann",
            "parent": "johanna",
            "sister_in_law": "christina",
            "uncle": "franz",
            "uncle": "heinz"
    }
\end{verbatim}
%% 
Note that we set the \texttt{duplicate\_keys=true} option because in the
\texttt{family\_obj} demo most of the properties (like \texttt{ancestor})
are multi-valued, which leads to repeated keys in JSON representation.
As we noted, this is allowed, but some applications do not support such
JSON expressions. If one needs to talk to such applications, simply don't
set the \texttt{duplicate\_keys=true} option and the above will represent
duplicate JSON keys using lists. For instance,
\texttt{"ancestor":["hermann","johanna","rita","wilhelm"]}.
Note, however, that without the \texttt{duplicate\_keys}
option the JSON encoding becomes
lossy, since we no longer can tell whether the original \ERGO
attribute \texttt{ancestor}
was multivalued (with each single value being a string)
or it was single-valued and the value was an ordered list.

Here we also note that the use of JSON API can often be simpler if one
recalls the very useful syntax of path expressions. For instance, the 3d
and 4th lines in the above query can be written much more succinctly as
%% 
\begin{verbatim}
   string(kati.object2json)[parse->pretty(?Res)]@\json
\end{verbatim}
%% 
If we try to encode the class \texttt{female} we get the following:
%% 
\begin{verbatim}
   string(kati.object2json)[parse->pretty(?Res)]@\json, writeln(?Res)@\io.
   {
    "\\self": "female",
    "\\sub": [
        "person"
    ],
    "type": "gender"
   }
\end{verbatim}
%% 
Note that \ERGO properties can be HiLog terms and so they cannot be
encoded simply as a string like \texttt{"parent"}. For instance,
%% 
\begin{verbatim}
   ?- insert{{a,b}:{c,d},d::k, k[|eee(123)->kkk|]}.
   ?- a[object2json -> ?Json]@\json,
      string(?Json)[parse->pretty(?Res)]@\json,
      writeln(?Res)@\io.

   {
    "\\self": "a",
    "\\isa": [
        "c",
        "d"
    ],
    "\\keyval": [
        {
            "functor": "eee",
            "arguments": [
                123
            ]
        },
        [
            "kkk"
        ]
    ]
   }
\end{verbatim}
%% 
Note that \texttt{eee(123) -> kkk} is a complex property that object
\texttt{a} inherits from class \texttt{k}. It is encoded as a JSON
keypair \texttt{"\bs\bs{}keyval"} : \emph{list} where the first element of
\emph{list} is the encoding of \texttt{eee(123)} and the second of
\texttt{"kkk"}.     

Now we are ready to present the different versions of the
\texttt{object2json} method.
%% 
\begin{itemize}
\item  \texttt{?Obj[object2json -> ?Json]@\bs{}json} --- take an object and
  return a
  Prolog atom that contains a JSON representation of the object's
  immediate superclasses and properties with respect to the \ERGO module
  where this call is made.
\item \texttt{?Obj[object2json(?Module) -> ?Json]@\bs{}json} --- as above, but
  the properties and the superclasses of \texttt{?Obj} are taken from the
  module \texttt{?Module}.
\item
  \texttt{?Obj[object2json(?Mod)(?keyFilter,?valFilter,?classFilter)->?Json]@\bs{}json}
  --- this version lets one to not only specify the module but also impose
  conditions on the properties of \texttt{?Obj}, on the superclasses, and
  on the property values that we want to see in the JSON representation.
  In the above, \texttt{(?Mod)} can be omitted and the current module will
  be used then. A \texttt{null} (or any
  other constant) condition means ``no filtering for that type of
  argument.''  Otherwise, the filters must be unary predicates or
  primitives. In the example below we use unary primitives
  \texttt{isnumber\{?\}} and \texttt{isatom\{?\}}.
\end{itemize}
%% 
First, we show what happens without filtering. It is an expansion of an
earlier example:
%% 
\begin{verbatim}
ergo> insert{{a,b}:c, c::{h,k}, h[|www->1|],k[|ppp->kk, eee(123)->kkk|]},
      string(a.object2json)[parse->pretty(?Res)]@\json, writeln(?Res)@\io.
{
    "\\self": "a",
    "\\isa": [
        "c"
    ],
    "ppp": [
        "kk"
    ],
    "www": [
        1
    ],
    "\\keyval": [
        {
            "functor": "eee",
            "arguments": [
                123
            ]
        },
        [
            "kkk"
        ]
    ]
}
\end{verbatim}
%% 
In contrast, the following query says that we want to see only the atomic
properties (so \texttt{eee(123)} will be omitted) and only such properties
whose values are numbers. No restrictions on superclasses is imposed:
%% 
\begin{verbatim}
ergo> string(a.object2json(isatomic{?},isnumber{?},null))[
                parse->pretty(?Res)
      ]@\json, writeln(?Res)@\io.
{
    "\\self": "a",
    "\\isa": [
        "c"
    ],
    "www": [
        1
    ]
}
\end{verbatim}
%% 
We see that the complex property \texttt{eee(123)->1} got dropped because
it is not atomic and the property \texttt{"ppp"} got dropped because its
values are not integers.

\paragraph{Recursive export.} Sometimes it is desirable to convert not just
an object, but an object together with its descendant objects---the ones
reachable from the object via its attributes---into a single JSON
structure. For instance, in our \texttt{family\_obj.flr}  example,
\texttt{kati} has an \texttt{ancestor}-descendant object \texttt{hermann},
which is also a  person-object that has its own JSON representation. We
might want to attach that representation to the \texttt{kati}-JSON
structure at the point where \texttt{"hermann"} is attached.
To enable such a \emph{recursive} export into JSON, one must set the
\texttt{recursive\_export} option by executing the following query:
%% 
\begin{verbatim}
   ?- set_option(recursive_export=true)@\json.
\end{verbatim}
%% 
We cannot show here the result of a recursive export for \texttt{kati}, as
the resulting structure is too big, but we will show a smaller example:
%% 
\begin{verbatim}
   ergo> insert{{a,b}:d, d::e, e::k ,k[|ppp->kk:d[prop1->abc,prop2->3], ppp->jj|]},
         string(a.object2json)[parse->pretty(?_Res)]@\json, writeln(?_Res)@\io.
    {
        "\\self": "a",
        "\\isa": [
            "d"
        ],
        "ppp": [
            {
                "\\self": "jj"
            },
            {
                "\\self": "kk",
                "\\isa": [
                    "d"
                ],
                "ppp": [
                    {
                        "\\self": "jj"
                    },
                    {
                        "\\self": "kk",
                        "\\isa": [
                            "d"
                        ]
                    }
                ],
                "prop1": [
                    {
                        "\\self": "abc"
                    }
                ],
                "prop2": [
                    {
                        "\\self": 3
                    }
                ]
            }
        ]
    }
\end{verbatim}
%% 
Here we see that \texttt{"kk"} (a \texttt{ppp}-descendant object of
\texttt{"a"}) is also JSON-expanded. Moreover, it is easy to see that
\texttt{kk[ppp->kk]} is true, which means that \texttt{kk} is a
\texttt{ppp}-descendant of itself. Thus, there is a cycle through
\texttt{kk} in the descendant-object relation and if we kept expanding
\texttt{kk} as we traverse the \texttt{ppp} attribute, the resulting JSON
term would be infinite. Therefore, as you can see, the second time we
encounter \texttt{"kk"} it is \emph{not} expanded and only its
isa-information is shown (the sub-information would have also been shown,
if it existed).





%%% Local Variables: 
%%% mode: latex
%%% TeX-master: "../ergo-packages"
%%% End: 
